\newcommand*{\No}{\textnumero}
\documentclass[russian, utf8, 12pt,pointsubsection,floatsubsection]{eskdtext}
\usepackage[russian]{babel} 

	\ESKDclassCode{ТП}
	\ESKDtitle{Автоматизированная система подачи заявления о выдаче водительских прав}
 	% для Первой спецификации
	\ESKDdocName{Техническое задание}
	\ESKDsignature{65698922.425120.022.ТЗ.01.01}
	\ESKDcolumnII{65698922.425120.022.ТЗ.01.01}
	\ESKDcolumnI{Программная документация. ООО «ЛюксКрафт». ТЗ. Ред.1}

	\ESKDgroup{ООО <<Опти-Софт>>}
	%\ESKDauthor{Эльвест К.В.}
	\ESKDtitleAgreedBy{Директор ООО <<Название>>}{Фамилия И.О.}
	\ESKDtitleDesignedBy{Зам. директора <<Название>>}{Фамилия И.О.}

	\ESKDtitleApprovedBy{Директор ООО «Название»}{Фамилия И.О.} 
	\ESKDtitleAgreedBy{Главный технолог ООО «Название»}{Фамилия И.О.} 


 	\ESKDdate{2020/09/30} 

% \title{Text}
% \author{ }
% \date{September 2021}

\begin{document}

	% титульный лист 
	\maketitle 

	% оглавление 
	\scriptsize
	\setcounter{tocdepth}{4}
	\tableofcontents
	\normalsize
	\newpage
	

\section{Общие сведения}


% В разделе «Общие сведения» указывают: 
% 1) полное наименование системы и ее условное обозначение; 
% 2) шифр темы или шифр (номер) договора; 
% 3) наименование предприятий (объединений) разработчика и заказчика (пользователя) системы и их реквизиты; 
% 4) перечень документов, на основании которых создается система, кем и когда утверждены эти документы; 
% 5) плановые сроки начала и окончания работы по созданию системы; 
% 6) сведения об источниках и порядке финансирования работ; 
% 7) порядок оформления и предъявления заказчику результатов работ по созданию системы (ее частей), по изготовлению и наладке отдельных средств (технических, программных, информационных) и программно-технических (программно-методических) комплексов системы. 

\subsection{Полное наименование системы и ее условное обозначение}
Полное наименование: Система автоматизации управления подачи заявлений о выдачи водительских прав "ГлавПрав".\\

Условное обозначение: "ГлавПрав"

\subsection{Номер договора (контракта)}
Номер договора: №1/17-11-22 от 17.11.2022

\subsection{Наименования организации-заказчика и организаций-участников работ}
Заказчик: МРЭО ГИБДД МВД по Республике Карелия; \\

Адрес: 180034, Республика Карелия, г. Петрозаводск, ул. Медьвежьегорская, д.3\\
Телефон/факс: 8(814)270-73-56 \\

Разработчики: студентка института математики и информационных технологий ПетрГУ, специальности «Прикладная математика и информатика», группы 22403: Першина О.В.  \\

 Адрес разработчика: г. Петрозаводск, ул. Петрова, д. 27

\subsection{Перечень документов, на основании которых создается система}
Работа выполняется на основании договоров №1/17-11-22 от 17.11.2022 между Заказчиком МРЭО ГИБДД МВД  и разработчиками ПетрГУ. Документы обяжут выполнять требования и пункты, описанные в договоре с обеих сторон.

\subsection{Плановые сроки начала и окончания работы по созданию системы}
Плановый срок начала работ по созданию системы приема заказов – 17 ноября 2022 года.
Плановый срок окончания работ по созданию системы приема заказа – 30 декабря 2022 года.

\subsection{Источники и порядок финансирования работ}
Источником	финансирования	является	бюджет	МРЭО ГИБДД МВД.
Порядок	финансирования	определяется	условиями	Договора
№1/17-11-22 от 17.11.2022

\subsection{Порядок оформления и предъявления заказчику результатов работ по созданию системы}
Разработчики по завершении работ по созданию системы предоставляет программу: АИС «ГлавПрав», осуществляет её установку на оборудование заказчика, а также предоставляет необходимую документацию. \\

Порядок	оформления	и предъявления результатов работ должен соответствовать требованиям комплекса стандартов и руководящих документов на автоматизированные системы: ГОСТ 34.201-89, ГОСТ 34.602-2020, ГОСТ 34.601-90.
Разработка проектных решений Системы, ее подсистем и (или) ее частей должно осуществляться в соответствии с данным Техническим заданием и исходными данными, предоставляемыми Заказчиком. В случае необходимости представители Исполнителя проводят обследование объекта автоматизации с участием представителей Заказчика.


\subsection{Перечень нормативно-технических документов, методических материалов, использованных при разработке ТЗ}
При разработке Системы и создании проектно-эксплуатационной документации Исполнитель должен руководствоваться следующими нормативными документами:
\begin{enumerate}
    \item Договор №1/17-11-22
    \item ГОСТ 2.105-95. Единая система конструкторской документации. Общие требования к текстовым документам
    \item ГОСТ 19.201-78. ТЕХНИЧЕСКОЕ ЗАДАНИЕ. ТРЕБОВАНИЯ К СОДЕРЖАНИЮ И ОФОРМЛЕНИЮ;
    \item ГОСТ	34.601-90.	Комплекс	стандартов	на	автоматизированные системы. Автоматизированные системы. Стадии создания;
    \item ГОСТ 34.602.2020 Техническое задание на создание автоматизированной системы.
\end{enumerate}

\subsection{Определения, обозначения и сокращения}
АИС - Автоматизированная Информационная Система
АСУ - Автоматизированная Система Управления


\section{НАЗНАЧЕНИЕ И ЦЕЛИ СОЗДАНИЯ СИСТЕМЫ}
% Раздел «Назначение и цели создания (развития) системы» состоит из подразделов: 
% 1) назначение системы; 
% 2) цели создания системы.
\subsection{Назначение системы}
Автоматизированная система "ГлавПрав" предназначена для управления оформлением и приёмом заявок на получение прав.

\subsection{Цели создания системы}
Основными целями внедрения системы являются:
\begin{itemize}
    \item Упрощение системы оформления заявок
    \item Перевод документооборота, ведущегося в бумажном виде, в электронный вид, упрощение и ускорение процесса оформления
    \item Повышение качества принятия управленческих решений за счёт оперативности представления и удобства отображения информации
    \item Уменьшения времени затрачиваемого на подачу и оформление заявки со стороны клиента, удобно предоставления требуемых документов
\end{itemize}


\section{ХАРАКТЕРИСТИКА ОБЪЕКТА АВТОМАТИЗАЦИИ}
% В разделе «Характеристики объекта автоматизации» приводят: 
% 1) краткие сведения об объекте автоматизации или ссылки на документы, содержащие такую информацию; 
% 2) сведения об условиях эксплуатации объекта автоматизации и характеристиках окружающей среды. 

% Примечание: Для САПР в разделе дополнительно приводят основные параметры и характеристики объектов проектирования. 

Характеристика объекта автоматизации изложена в документе описания предметной области «Автоматизированная система подачи заявления о выдаче водительских прав». А также в документе «Исследование организации», разработанном в процессе обследования предприятия заказчика.



\section{ТРЕБОВАНИЯ К СИСТЕМЕ}
% Раздел «Требования к системе» состоит из следующих подразделов: 
% 1) требования к системе в целом; 
% 2) требования к функциям (задачам), выполняемым системой; 
% 3) требования к видам обеспечения.
% Состав требований к системе, включаемых в данный раздел ТЗ на АС, устанавливают в зависимости от вида, назначения, специфических особенностей и условий функционирования конкретной системы. В каждом подразделе приводят ссылки на действующие НТД, определяющие требования к системам соответствующего вида. 
\subsection{Требования к системе в целом}
% В подразделе «Требования к системе в целом» указывают: 
% –	требования к структуре и функционированию системы; 
% –	требования к численности и квалификации персонала системы и режиму его работы; 
% –	показатели назначения; 
% –	требования к надежности; 
% –	требования безопасности; 
% –	требования к эргономике и технической эстетике; 
% –	требования к транспортабельности для подвижных АС; 
% –	требования к эксплуатации, техническому обслуживанию, ремонту и хранению компонентов системы; 
% –	требования к защите информации от несанкционированного доступа; 
% –	требования по сохранности информации при авариях; 
% –	требования к защите от влияния внешних воздействий; 
% –	требования к патентной чистоте; 
% –	требования по стандартизации и унификации;
% –	дополнительные требования. 
\point{Требования к структуре и функционированию системы}
% В требованиях к структуре и функционированию системы приводят: 
% 1) перечень подсистем, их назначение и основные характеристики, требования к числу уровней иерархии и степени централизации системы; 
% 2) требования к способам и средствам связи для информационного обмена между компонентами системы; 
% 3) требования к характеристикам взаимосвязей создаваемой системы со смежными системами, требования к ее совместимости, в том числе указания о способах обмена информацией (автоматически, пересылкой документов, по телефону и т. п.); 
% 4) требования к режимам функционирования системы; 
% 5) требования по диагностированию системы; 
% 6) перспективы развития, модернизации системы.

% Данной раздел можно разбить на подразделы:
\subpoint{Перечень подсистем, их назначение и основные характеристики}

В состав автоматизированной системы "ГлавПрав" должны входить следующие подсистемы:
\begin{itemize}
    \item Подсистема формирования заявления
    \item Подсистема обработки заявлений
    \item Подсистема отслеживания статуса заявления
\end{itemize}

Подсистема формирования заявления предназначена для автоматизации подачи документов и заявки на получение прав. \\
Подсистема обработки заявлений предназначена для автоматизации процесса приёма заявлений от клиентов со стороны компании.\\
Подсистема отслеживания статуса заявления предназначена для клиентского контроля над процессом приёма заявления и уведомления клиента о необходимости посещения подразделения для получения прав.

\subpoint{Требования к способам и средствам связи для информационного обмена между компонентами системы}

Компоненты автоматизированной системы должны связываться друг с другом в обоюдной порядке для получения, обмена и обработки информации о поданных заявках. 

\point{Требования к численности и квалификации персонала системы}\\
% В требованиях к численности и квалификации персонала на АС приводят: 
% –	требования к численности персонала (пользователей) АС; 
% –	требования к квалификации персонала, порядку его подготовки и контроля знаний и навыков; 
% –	требуемый режим работы персонала АС. 
Пользователи системы, менеджеры по приёму заявок, должны иметь опыт работы с персональным компьютером на базе операционных систем Microsoft Windows на уровне квалифицированного пользователя и свободно осуществлять базовые операции в стандартных Windows.
\point{Показатели назначения}\\
% В требованиях к показателям назначения АС приводят значения параметров, характеризующие степень соответствия системы ее назначению. 
% Для АСУ указывают: 
% –	степень приспособляемости системы к изменению процессов и методов управления, к отклонениям параметров объекта управления; 
% –	допустимые пределы модернизации и развития системы; 
% –	вероятностно-временные характеристики, при которых сохраняется целевое назначение системы. 
Целевое назначение системы должно сохраняться на протяжении всего срока эксплуатации компании. Срок эксплуатации определяется сроком устойчивой работы аппаратных средств, своевременным проведением работ по замене (обновлению) аппаратных средств, по сопровождению программного обеспечения системы и его модернизации.\\
Время выполнения запросов информации в АИС определяется на стадии проектирования системы.\\
В АИС должны быть обеспечены возможности по созданию, добавлению, изменению и удалению полей данных в пользовательском интерфейсе.
Прочие показатели назначения АИС разрабатываются после проведения предварительного обследования.

\point{Требования к надежности}\\
% В требования к надежности включают: 
% 1) состав и количественные значения показателей надежности для системы в целом или ее подсистем; 
% 2) перечень аварийных ситуаций, по которым должны быть регламентированы требования к надежности, и значения соответствующих показателей; 
% 3) требования к надежности технических средств и программного обеспечения; 
% 4) требования к методам оценки и контроля показателей надежности на разных стадиях создания системы в соответствии с действующими нормативно-техническими документами. 
Система должна сохранять работоспособность и обеспечивать восстановление своих функций при возникновении следующих внештатных ситуаций:
\begin{itemize}
    \item при сбоях в системе электроснабжения аппаратной части, приводящих к перезагрузке ОС, восстановление программы должно происходить после перезапуска ОС и запуска исполняемого файла системы;
    \item при ошибках в работе аппаратных средств (кроме носителей данных и программ) восстановление функции системы возлагается на ОС;
    \item при ошибках, связанных с программным обеспечением (ОС и драйверы устройств), восстановление работоспособности возлагается на ОС.
\end{itemize}

Для защиты аппаратуры от бросков напряжения и коммутационных
помех должны применяться сетевые фильтры.\\
Время восстановления работоспособности прикладного ПО АИС при любых сбоях и отказах не должно превышать одного рабочего дня.\\
Уровень надежности должен достигаться согласованным применением	организационных,	организационно-технических мероприятий и программно-аппаратных средств.\\

Надежность должна обеспечиваться за счет:
\begin{itemize}
    \item применения технических средств, системного и базового программного обеспечения, соответствующих классу решаемых задач.
    \item Должно осуществляться разграничение прав доступа к системе.
    \item В АИС должна быть обеспечена возможность восстановления данных с внешнего накопителя после восстановления активного накопителя.
    \item своевременного выполнения процессов администрирования Системы.
    \item соблюдения правил эксплуатации и технического обслуживания программно-аппаратных средств.
\end{itemize}


\point{Требования к безопасности}\\
% В требования по безопасности включают требования по обеспечению безопасности при монтаже, наладке, эксплуатации, обслуживании и ремонте технических средств системы (защита от воздействий электрического тока, электромагнитных полей, акустических шумов и т. п.), по допустимым уровням освещенности, вибрационных и шумовых нагрузок.

Сотрудники Системы при работе с компьютером, либо другим вычислительным устройством, соединенным с сетью должен соблюдать технику безопасности.\\

Все внешние элементы технических средств системы, находящиеся под напряжением, должны иметь защиту от случайного прикосновения, а сами технические средства иметь зануление или защитное заземление в соответствии с ГОСТ 12.1.030-81 и ПУЭ.\\

Общие требования пожарной безопасности должны соответствовать нормам на бытовое электрооборудование. В случае возгорания не должно выделяться ядовитых газов и дымов. После снятия электропитания должно быть допустимо применение любых средств пожаротушения.


\point{Требования к эргономике и технической эстетике}\\
% В требования по эргономике и технической эстетике включают показатели АС, задающие необходимое качество взаимодействия человека с машиной и комфортность условий работы персонала.

Работа пользователя с системой должна осуществляться посредством графического интерфейса. Графический интерфейс должен обладать следующими свойствами:
\begin{itemize}
    \item должен соответствовать современным эргономическим требованиям и обеспечивать удобный доступ к основным функциям и операциям системы;
    \item управление системой должно осуществляться с помощью набора экранных меню, кнопок, значков и т. п. элементов;
    \item должен быть понятным и удобным для использования;
взаимодействие пользователя с интерфейсом должно осуществляться посредством компьютера с доступом к интернету;
    \item все надписи экранных форм, а также сообщения, выдаваемые
пользователю (кроме системных сообщений) должны быть на русском языке;
    \item должен быть выполнен в едином графическом дизайне, приятном для восприятия;
\end{itemize}

\point{Требования к эксплуатации, техническому обслуживанию, ремонту и хранению компонентов системы}\\
% В требования к эксплуатации, техническому обслуживанию, ремонту и хранению включают: 
% 1) условия и регламент (режим) эксплуатации, которые должны обеспечивать использование технических средств (ТС) системы с заданными техническими показателями, в том числе виды и периодичность обслуживания ТС системы или допустимость работы без обслуживания; 
% 2) предварительные требования к допустимым площадям для размещения персонала и ТС системы, к параметрам сетей энергоснабжения и т. п.; 
% 3) требования по количеству, квалификации обслуживающего персонала и режимам его работы; 
% 4) требования к составу, размещению и условиям хранения комплекта запасных изделий и приборов; 
% 5) требования к регламенту обслуживания. 

Система должна обеспечивать круглосуточный режим работы.

Условия     эксплуатации     и	периодичность обслуживания технических средств системы должны соответствовать требованиям по эксплуатации, техническому обслуживанию, ремонту и хранению, изложенным в документации завода-изготовителя (производителя).

Периодическое техническое обслуживание используемых технических средств должно проводиться в соответствии с требованиями технической документации изготовителей, но не реже одного раза в год. 

Для электропитания технических средств должна быть предусмотрена трехфазная четырехпроводная сеть с глухо заземленной нейтралью 380/220 В (+10-15)\% частотой 50 Гц (+1-1) Гц. Каждое техническое средство запитывается однофазным напряжением 220 В частотой 50 Гц через сетевые розетки с заземляющим контактом.

Для обеспечения выполнения требований по надежности должен быть создан комплект запасных изделий и приборов (ЗИП).

Размещение помещений и их оборудование должны исключать возможность бесконтрольного проникновения в них посторонних лиц и обеспечивать сохранность находящихся в этих помещениях конфиденциальных документов и технических средств.

Размещение оборудования, технических средств должно соответствовать требованиям техники безопасности, санитарным нормам и требованиям пожарной безопасности.

Все пользователи системы должны соблюдать правила эксплуатации электронной вычислительной техники.

Квалификация персонала и его подготовка должны соответствовать технической документации.


\point{Требования к защите информации от несанкционированного доступа}
% В требования к защите информации от несанкционированного доступа включают требования, установленные в НТД, действующей в отрасли (ведомстве) заказчика.

Компоненты подсистемы защиты должны обеспечивать:
\begin{enumerate}
    \item идентификацию пользователя;
\item проверку полномочий пользователя при работе с системой;
\item разграничение доступа пользователей на уровне задач и информационных массивов
\end{enumerate}

Защищённая часть системы должна использовать "слепые" пароли (при наборе пароля его символы не показываются на экране либо заменяются одним типом символов; количество символов не соответствует длине пароля).

Защищённая часть системы должна автоматически блокировать сессии пользователей и приложений по заранее заданным временам отсутствия активности со стороны пользователей и приложений.

Защищённая часть системы должна использовать многоуровневую систему защиты.

Защищённая часть системы должна быть отделена от незащищенной части системы межсетевым экраном.

В системе должны быть предусмотрены механизмы исправления неверно проведенных операций. При этом должна соблюдаться принятая Заказчиком технология, предусматривающая подобные случаи, а также обеспечиваться регистрация исправительных действий в соответствующих журналах для последующего контроля.


\point{Требования по сохранности информации при авариях}
% В требованиях по сохранности информации приводят перечень событий: аварий, отказов технических средств (в том числе - потеря питания) и т. п., при которых должна быть обеспечена сохранность информации в системе.

Используемые аппаратные и системные платформы должны обеспечивать сохранность и целостность информации в системе при полном или частичном отключении электропитания, аварии сетей телекоммуникации, полном или частичном отказе технических средств системы.

В системе должны быть предусмотрены меры, обеспечивающие целостность данных в случае отказа аппаратных средств или программного обеспечения.

Сохранность информации в системе должна быть обеспечена при:
\begin{enumerate}
    \item отключении электропитания;

\item отказе компьютера, на котором работает программа;

\item временном отказе линий связи.
\end{enumerate}

Программное обеспечение должно восстанавливать свое функционирование при корректном перезапуске аппаратных средств. Должна быть предусмотрена возможность организации автоматического и (или) ручного резервного копирования данных системы средствами системного и базового программного обеспечения (ОС, СУБД), входящего в состав программно-технического комплекса заказчика.


\point{Требования к защите от влияния внешних воздействий}
% В требованиях к средствам защиты от внешних воздействий приводят: 
% 1) требования к радиоэлектронной защите средств АС; 
% 2) требования по стойкости, устойчивости и прочности к внешним воздействиям (среде применения). 

Защита от влияния внешних воздействий должна обеспечиваться средствами программно-технического комплекса Заказчика.

\point{Требования к патентной частоте}
% В требованиях по патентной чистоте указывают перечень стран, в отношении которых должна быть обеспечена патентная чистота системы и ее частей.

Установка системы в целом, как и установка отдельных частей системы не должна предъявлять дополнительных требований к покупке лицензий на программное обеспечение сторонних производителей.

\point{Требования по стандартизации и унификации}
% В требования к стандартизации и унификации включают: показатели, устанавливающие требуемую степень использования стандартных, унифицированных методов реализации функций (задач) системы, поставляемых программных средств, типовых математических методов и моделей, типовых проектных решений, унифицированных форм управленческих документов, установленных ГОСТ 6.10.1, общесоюзных классификаторов технико-экономической информации и классификаторов других категорий в соответствии с областью их применения, требования к использованию типовых автоматизированных рабочих мест, компонентов и комплексов.

Разрабатываемая система должна соответствовать:
\begin{enumerate}
    \item ГОСТу 34.601-90 «Комплекс стандартов на автоматизированные системы. Автоматизированные системы. Стадии создания»;
    \item ГОСТу 34.201-89 «Информационная технология. Комплекс стандартов на автоматизированные системы. Виды, комплексность и обозначение документов при создании автоматизированных систем»;
    \item РДу 50-34.698-90 «Автоматизированные системы. Требования к содержанию документов».
    \item ГОСТу 34.603-92 «Информационная технология. Виды испытаний автоматизированных систем».
\end{enumerate}

В    системе     должны     использоваться    (при    необходимости)
общероссийские классификаторы и единые классификаторы и словари для различных видов алфавитно-цифровой и текстовой информации.


\point{Дополнительные требования}
% В дополнительные требования включают: 
% 1) требования к оснащению системы устройствами для обучения персонала (тренажерами, другими устройствами аналогичного назначения) и документацией на них; 
% 2) требования к сервисной аппаратуре, стендам для проверки элементов системы; 
% 3) требования к системе, связанные с особыми условиями эксплуатации; 
% 4) специальные требования по усмотрению разработчика или заказчика системы. 
% 5.2 Требования к функциям (задачам), выполняемым системой
% В подразделе «Требование к функциям (задачам)», выполняемым системой, приводят: 
% 1) по каждой подсистеме перечень функций, задач или их комплексов (в том числе обеспечивающих взаимодействие частей системы), подлежащих автоматизации; 
% при создании системы в две или более очереди - перечень функциональных подсистем, отдельных функций или задач, вводимых в действие в 1-й и последующих очередях; 
% 2) временной регламент реализации каждой функции, задачи (или комплекса задач); 
% 3) требования к качеству реализации каждой функции (задачи или комплекса задач), к форме представления выходной информации, характеристики необходимой точности и времени выполнения, требования одновременности выполнения группы функций, достоверности выдачи результатов; 
% 4) перечень и критерии отказов для каждой функции, по которой задаются требования по надежности.

Дополнительных требований не предъявлено.

\subsection{Требования к функциям (задачам), выполняемым системой}
% В подразделе «Требование к функциям (задачам)», выполняемым системой, приводят: 
% 1) по каждой подсистеме перечень функций, задач или их комплексов (в том числе обеспечивающих взаимодействие частей системы), подлежащих автоматизации; 
% при создании системы в две или более очереди - перечень функциональных подсистем, отдельных функций или задач, вводимых в действие в 1-й и последующих очередях; 
% 2) временной регламент реализации каждой функции, задачи (или комплекса задач); 
% 3) требования к качеству реализации каждой функции (задачи или комплекса задач), к форме представления выходной информации, характеристики необходимой точности и времени выполнения, требования одновременности выполнения группы функций, достоверности выдачи результатов; 
% 4) перечень и критерии отказов для каждой функции, по которой задаются требования по надежности.

Автоматизированная система подразделена на следующие подсистемы:
\begin{enumerate}
    \item Подсистема формирования заявления
    \item Подсистема обработки заявления
    \item Подсистема отслеживания заявлений
\end{enumerate}

\begin{enumerate}
    \item Подсистема формирования заявления имеет следующие функции:
    \begin{itemize}
        \item Получение данных клиента для заполнения заявки
        \item Отправка заявления в подразделение
    \end{itemize}
    \item Подсистема обработки заявления имеет следующие функции:
    \begin{itemize}
        \item Получение заявки от клиента
        \item Оформление заявки клиента
        \item Подтверждение закрытия заявки
    \end{itemize}
    \item Подсистема отслеживания заявлений имеет следующие функции:
    \begin{itemize}
        \item Отслеживание статуса заявления
    \end{itemize}
\end{enumerate}

\subsubsection{Описание требований к функции ''Получение данных клиента для заполнения заявки''}
Функция "Получение данных клиента для заполнения заявки" должна собирать у клиента необходимые данные для заполнения заявки на получение прав.

Для этого необходимо выполнить следующие подзадачи:
\begin{itemize}
    \item Получение необходимых документов
    \item Проверка наличия всех требуемых данных
    \item Формирование заявления для отправки в подразделение
\end{itemize}

\subsubsection{Описание требований к функции ''Отправка заявления в подразделение''}
Функция "Отправка заявления в подразделение" должна отправлять сформированное заявление от клиента в выбранное подразделение для обработки.

Для этого необходимо выполнить следующие подзадачи:
\begin{itemize}
    \item Получение информации о выборе подразделения
    \item Проверка возможности подразделения принять заявление
    \item Отправка сформированного заявления в подразделение
\end{itemize}

\subsubsection{Описание требований к функции ''Получение заявки от клиента''}
Функция "Получение заявки от клиента" должна принимать данные клиента и передавать их работнику системы.

Для этого необходимо выполнить следующие подзадачи:
\begin{itemize}
    \item Приём данных со стороны клиента 
    \item Добавление заявления в список входящих заявок
    \item Предоставление заявления сотруднику
\end{itemize}


\subsubsection{Описание требований к функции ''Оформление заявки клиента''}
Функция "Оформление заявки клиента" должна с помощью сотрудника оформлять заявление клиента на получение прав.

Для этого необходимо выполнить следующие подзадачи:
\begin{itemize}
    \item Проверка заявления на корректность
    \item Формирование документа на получение прав
    \item Передача сформированного документа на закрытие
\end{itemize}

\subsubsection{Описание требований к функции ''Подтверждение закрытия заявки''}
Функция "Подтверждение закрытия заявки" должна предоставлять сотруднику сформированные документы для создания приглашения клиенту на получение прав.

Для этого необходимо выполнить следующие подзадачи:
\begin{itemize}
    \item Предоставлять сотруднику документ о получении прав
    \item Назначение даты приглашения клиента на получение прав
    \item Перевод документа из списка текущих заявок
\end{itemize}

\subsubsection{Описание требований к функции ''Отслеживание статуса заявления''}
Функция "Отслеживание статуса заявления" должна предоставлять клиенту информацию о текущем состоянии его заявления.

Для этого необходимо выполнить следующие подзадачи:
\begin{itemize}
    \item Предоставление информации о стадии рассмотрения заявки
    \item Уведомление клиента о завершении обработки заявления
    \item Предоставление даты личного визита в подразделения по окончанию оформления заявления
\end{itemize}



\subsection{Требования к видам обеспечения}
% В подразделе «Требования к видам обеспечения» в зависимости от вида системы приводят требования к математическому, информационному, лингвистическому, программному, техническому, метрологическому, организационному, методическому и другие видам обеспечения системы.
\subsubsection{Требования к математическому обеспечению системы}
% Для математического обеспечения системы приводят требования к составу, области применения (ограничения) и способам, использования в системе математических методов и моделей, типовых алгоритмов и алгоритмов, подлежащих разработке.
Математические методы и алгоритмы, используемые для шифрования/дешифрования данных, а также программное обеспечение, реализующее их, должны быть сертифицированы уполномоченными организациями для использования в государственных органах Российской Федерации.
\subsubsection{Требования информационному обеспечению системы}
% Для информационного обеспечения системы приводят требования: 
% 1) к составу, структуре и способам организации данных в системе; 
% 2) к информационному обмену между компонентами системы; 
% 3) к информационной совместимости со смежными системами; 
% 4) по использованию общесоюзных и зарегистрированных республиканских, отраслевых классификаторов, унифицированных документов и классификаторов, действующих на данном предприятии; 
% 5) по применению систем управления базами данных; 
% 6) к структуре процесса сбора, обработки, передачи данных в системе и представлению данных; 
% 7) к защите данных от разрушений при авариях и сбоях в электропитании системы; 
% 8) к контролю, хранению, обновлению и восстановлению данных; 
% 9) к процедуре придания юридической силы документам, продуцируемым техническими средствами АС (в соответствии с ГОСТ 6.10.4).
Состав, структура и способы организации данных в системе должны быть определены на этапе технического проектирования.

Уровень хранения данных в системе должен быть построен на основе современных реляционных или объектно-реляционных СУБД. Для обеспечения целостности данных должны использоваться встроенные механизмы СУБД.

Средства СУБД, а также средства используемых операционных систем должны обеспечивать документирование и протоколирование обрабатываемой в системе информации.

Структура базы данных должна поддерживать кодирование хранимой и обрабатываемой информации в соответствии с общероссийскими классификаторами (там, где они применимы).

Доступ к данным должен быть предоставлен только авторизованным пользователям с учетом их служебных полномочий, а также с учетом категории запрашиваемой информации.

Структура базы данных должна быть организована рациональным способом, исключающим единовременную полную выгрузку информации, содержащейся в базе данных системы.

Технические средства, обеспечивающие хранение информации, должны использовать современные технологии, позволяющие обеспечить повышенную надежность хранения данных и оперативную замену оборудования (распределенная избыточная запись/считывание данных; зеркалирование; независимые дисковые массивы; кластеризация).

В состав системы должна входить специализированная подсистема резервного копирования и восстановления данных.

При проектировании и развертывании системы необходимо рассмотреть возможность использования накопленной информации из уже функционирующих информационных систем. Перечень функционирующих информационных систем приведен в разделе 3 настоящего документа.


\subsubsection{Требования к лингвистическому обеспечению системы}
% Для лингвистического обеспечения системы приводят требования к применению в системе языков программирования высокого уровня, языков взаимодействия пользователей и технических средств системы, а также требования к кодированию и декодированию данных, к языкам ввода-вывода данных, языкам манипулирования данными, средствам описания предметной области (объекта автоматизации), к способам организации диалога.
Все прикладное программное обеспечение системы для организации взаимодействия с пользователем должно использовать русский язык.


\subsubsection{Требования к программному обеспечению системы}
% Для программного обеспечения системы приводят перечень покупных программных средств, а также требования: 
% 1) к независимости программных средств от используемых СВТ и операционной среды; 
% 2) к качеству программных средств, а также к способам его обеспечения и контроля; 
% 3) по необходимости согласования вновь разрабатываемых программных средств с фондом алгоритмов и программ.
При проектировании и разработке системы необходимо максимально эффективным образом использовать ранее закупленное программное обеспечение, как серверное, так и для рабочих станций.

Используемое при разработке программное обеспечение и библиотеки программных кодов должны иметь широкое распространение, быть общедоступными и использоваться в промышленных масштабах. Базовой программной платформой должна являться операционная система MS Windows.


\subsubsection{Требования к техническому обеспечению}
% Для технического обеспечения системы приводят требования: 
% 1) к видам технических средств, в том числе к видам комплексов технических средств, программно-технических комплексов и других комплектующих изделий, допустимых к использованию в системе; 
% 2) к функциональным, конструктивным и эксплуатационным характеристикам средств технического обеспечения системы.
Техническое обеспечение системы должно максимально и наиболее эффективным образом использовать существующие в организации технические средства. В состав комплекса должны входить следующие технические средства:
\begin{itemize}
    \item Сервер БД;
    \item ПК администраторов
\end{itemize}

Сервер и рабочие станции должны быть объединены одной локальной сетью с пропускной способностью не менее 25 Мбит/с.

Требования к техническим характеристикам сервера БД:
\begin{itemize}
    \item Процессор – Intel Xeon 3 ГГц;

\item Операционная система – Microsoft Windows Server 2012 или выше;

\item Объем оперативной памяти – 8 Гб;

\item Дисковая подсистема – 2 х 150 Гб;

\item Сетевой адаптер – 25 Мбит/с.
\end{itemize}

Требования к техническим характеристикам рабочих станций:
\begin{itemize}
    \item Процессор – Intel Core i3 1,2 ГГц;

\item Объем оперативной памяти – 1 Гб;

\item Объем жесткого диска – 80 Гб;

\item Операционная система – Windows 7/10;

\item Сетевой адаптер – 25 Мбит/с.

\item Дисковая подсистема – 2 х 150 Гб;

\item Сетевой адаптер – 25 Мбит/с.
\end{itemize}


\subsubsection{Требования к метрологическому обеспечению}
% В требованиях к метрологическому обеспечению приводят: 
% 1) предварительный перечень измерительных каналов; 
% 2) требования к точности измерений параметров и (или) к метрологическим характеристикам измерительных каналов; 
% 3) требования к метрологической совместимости технических средств системы; 
% 4) перечень управляющих и вычислительных каналов системы, для которых необходимо оценивать точностные характеристики; 
% 5) требования к метрологическому обеспечению технических и программных средств, входящих в состав измерительных каналов системы, средств, встроенного контроля, метрологической пригодности измерительных каналов и средств измерений, используемых при наладке и испытаниях системы; 
% 6) вид метрологической аттестации (государственная или ведомственная) с указанием порядка ее выполнения и организаций, проводящих аттестацию. 
Требования к метрологическому обеспечению не предъявляются.

\subsubsection{Требования к организационному обеспечению}
%  Для организационного обеспечения приводят требования: 
% 1) к структуре и функциям подразделений, участвующих в функционировании системы или обеспечивающих эксплуатацию; 
% 2) к организации функционирования системы и порядку взаимодействия персонала АС и персонала объекта автоматизации; 
% 3) к защите от ошибочных действий персонала системы.
Организационное обеспечение системы должно быть достаточным для эффективного выполнения персоналом возложенных на него обязанностей при осуществлении автоматизированных и связанных с ними неавтоматизированных функций системы.

Заказчиком должны быть определены должностные лица, ответственные за: обработку информации АИС, администрирование АИС, обеспечение безопасности информации АИС, управление работой персонала по обслуживанию АИС.

К работе с системой должны допускаться сотрудники, имеющие навыки работы на персональном компьютере, ознакомленные с правилами эксплуатации и прошедшие обучение работе с системой.


\subsubsection{Требования к методическому обеспечению}
% Для методического обеспечения САПР приводят требования к составу нормативно-технической документации системы (перечень применяемых при ее функционировании стандартов, нормативов, методик и т. п.).
 состав нормативно-правового и методического обеспечения системы должны входить следующие законодательные акты, стандарты и нормативы:
\begin{enumerate}
    \item Федеральный закон "Об информации, информационных технологиях и о защите информации" от 27.07.2006 N 149-ФЗ;
    \item Устав компании.

\end{enumerate}

\section{СОСТАВ И СОДЕРЖАНИЕ РАБОТ ПО СОЗДАНИЮ (РАЗВИТИЮ) СИСТЕМЫ}
% Раздел «Состав и содержание работ по созданию (развитию) системы» должен содержать перечень стадий и этапов работ по созданию системы в соответствии с ГОСТ 24.601, сроки их выполнения, перечень организаций - исполнителей работ, ссылки на документы, подтверждающие согласие этих организаций на участие в создании системы, или запись, определяющую ответственного (заказчик или разработчик) за проведение этих работ. 
% В данном разделе также приводят: 
% 1) перечень документов, по ГОСТ 34.201-89, предъявляемых по окончании соответствующих стадий и этапов работ; 
% 2) вид и порядок проведения экспертизы технической документации (стадия, этап, объем проверяемой документации, организация-эксперт); 
% 3) программу работ, направленных на обеспечение требуемого уровня надежности разрабатываемой системы (при необходимости); 
% 4) перечень работ по метрологическому обеспечению на всех стадиях создания системы с указанием их сроков выполнения и организаций-исполнителей (при необходимости). 
\begin{tabular}{|c|p{8cm}|p{6cm}|}
    \hline
   Этап & Содержание работ & Результат работ \\
   \hline
    1 & Разработка рабочей документации  АИС «ГлавПрав». & Рабочая документация АИС «ГлавПрав».\\
    \hline
    2 & Создание подсистем формирования оценочного листа, формирования сопроводительных актов, формирования отчётности. & Программное обеспечение указанных подсистем.\\
    \hline
    3 & Тестирование и отладка программного обеспечения АИС «ГлавПрав». & Отчёт о тестировании АИС «ГлавПрав», программное обеспечение АИС «ГлавПрав».\\
    \hline
    4 & Разработка руководства пользователя АИС «ГлавПрав». & Руководство пользователя АИС «ГлавПрав».\\
    \hline
    5 & Приёмо-сдаточные испытания АИС «ГлавПрав». & Акт приемочной комиссии АИС «ГлавПрав».\\
    \hline

\end{tabular}


\section{ПОРЯДОК КОНТРОЛЯ И ПРИЕМКИ СИСТЕМЫ}
% В разделе «Порядок контроля и приемки системы» указывают: 
% 1) виды, состав, объем и методы испытаний системы и ее составных частей (виды испытаний в соответствии с действующими нормами, распространяющимися на разрабатываемую систему); 
% 2) общие требования к приемке работ по стадиям (перечень участвующих предприятий и организаций, место и сроки проведения), порядок согласования и утверждения приемочной документации; 
% З) статус приемочной комиссии (государственная, межведомственная, ведомственная).
\subsection{Виды, состав, объем и методы испытаний системы}
Виды, состав, объем, и методы испытаний подсистемы должны быть изложены в программе и методике испытаний АИС «ГлавПрав», разрабатываемой в составе рабочей документации.
\subsection{Общие требования к приемке работ по стадиям}
Сдача-приемка осуществляется комиссией, в состав которой входят представители заказчика и исполнителя. По результатам приемки подписывается акт приемочной комиссии. Все создаваемые в рамках настоящей работы программные изделия (за исключением покупных) передаются заказчику как в виде готовых модулей, так и в виде исходных кодов, предоставляемых в электронной форме на стандартном машинном носителе (например, на flash-носителе).
\subsection{Статус приемочной комиссии}
Статус приемочной комиссии определяется заказчиком до проведения испытаний.

\section{ТРЕБОВАНИЯ К СОСТАВУ И СОДЕРЖАНИЮ РАБОТ ПО ПОДГОТОВКЕ ОБЪЕКТА АВТОМАТИЗАЦИИ К ВВОДУ СИСТЕМЫ В ДЕЙСТВИЕ}
% В разделе «Требования к составу и содержанию работ по подготовке объекта автоматизации к вводу системы в действие» необходимо привести перечень основных мероприятий и их исполнителей, которые следует выполнить при подготовке объекта автоматизации к вводу АС в действие. 
% В перечень основных мероприятий включают: 
% 1) приведение поступающей в систему информации (в соответствии с требованиями к информационному и лингвистическому обеспечению) к виду, пригодному для обработки с помощью ЭВМ; 
% 2) изменения, которые необходимо осуществить в объекте автоматизации; 
% 3) создание условий функционирования объекта автоматизации, при которых гарантируется соответствие создаваемой системы требованиям, содержащимся в ТЗ; 
% 4) создание необходимых для функционирования системы подразделений и служб; 
% 5) сроки и порядок комплектования штатов и обучения персонала. 
% Например, для АСУ приводят: 
% изменения применяемых методов управления; 
% создание условий для работы компонентов АСУ, при которых гарантируется соответствие системы требованиям, содержащимся в ТЗ. 
В ходе выполнения проекта на объекте автоматизации требуется выполнить работы по подготовке к вводу системы в действие.

При подготовке к вводу в эксплуатацию АИС ООО «Рыботорговая сеть» заказчик должен обеспечить выполнение следующих работ:
\begin{enumerate}
    \item Определить подразделение и ответственных должностных лиц, ответственных за внедрение и проведение опытной эксплуатации АИС «ГлавПрав»;
    \item Обеспечить присутствие пользователей на обучении работе с системой, проводимом исполнителем;
    \item Обеспечить соответствие помещений и рабочих мест пользователей системы с требованиями, изложенными в настоящем документе;
    \item Обеспечить выполнение требований, предъявляемых к программно-техническим средствам, на которых должно быть развернуто программное обеспечение АИС «ГлавПрав»;
    \item Совместно с исполнителем подготовить план развертывания системы на технических средствах заказчика;
    \item Провести опытную эксплуатацию АИС «ГлавПрав».
\end{enumerate} 

Требования к составу и содержанию работ по подготовке объекта автоматизации к вводу системы в действие, включая перечень основных мероприятий и их исполнителей должны быть уточнены на стадии подготовки рабочей документации и по результатам опытной эксплуатации.


\section{ТРЕБОВАНИЯ К ДОКУМЕНТИРОВАНИЮ}
% В разделе «Требования к документированию» приводят: 
% 1) согласованный разработчиком и Заказчиком системы перечень подлежащих разработке комплектов и видов документов, соответствующих требованиям ГОСТ 34.201-89 и НТД отрасли заказчика; перечень документов, выпускаемых на машинных носителях; требования к микрофильмированию документации; 
% 2) требования по документированию комплектующих элементов межотраслевого применения в соответствии с требованиями ЕСКД и ЕСПД; 
% 3) при отсутствии государственных стандартов, определяющих требования к документированию элементов системы, дополнительно включают требования к составу и содержанию таких документов.
Документы должны быть представлены в бумажном виде (оригинал) и на носителе (копия). Исходные тексты программ - только на носителе (оригинал).
Все документы должны быть оформлены на русском языке. Состав документов на общее программное обеспечение, поставляемое в составе АИС, должен соответствовать комплекту поставки компании - изготовителя.

Подлежащие разработке документы:
\begin{enumerate}
    \item Документ	«Описание	предметной	области»	согласно	со	стандартом оформления документации ГОСТ 2.105-95
\item Документ «Техническое задание» согласно ГОСТ 34.602.2020
\item Документ «Руководство пользователя» согласно РД 50-34.698-90 п.3.4.
\item Документ «Программа и методики испытаний» согласно ГОСТ 19.301-7
\item Документа «Паспорт» согласно ГОСТ 50-34.698-90 п.2.8.
\end{enumerate}

К видам программной документации относят документы, содержащие сведения, необходимые для разработки, изготовления, сопровождения и эксплуатации программ:
\begin{enumerate}
    \item Спецификация (состав программы и документации на нее)
\item Ведомость	держателей	подлинников	(перечень	предприятий,	на которых хранят подлинники программных документов)
\item Текст программы (запись программы с необходимыми комментариями)
\item Описание	программы	(сведения	о	логической	структуре	и функционировании программы)
\item Программа и методика испытаний (требования, подлежащие проверке при испытании программы, а также порядок и методы их контроля)
\item Техническое задание
\item Пояснительная	записка	(схема	алгоритма	функционирования программы, а также обоснование принятых технических решений)
\item Эксплуатационные	документы	(сведения	для	обеспечения функционирования и эксплуатации программы)

\end{enumerate}


\section{ИСТОЧНИКИ РАЗРАБОТКИ}
% В разделе «Источники разработки» должны быть перечислены документы и информационные материалы (технико-экономическое обоснование, отчеты о законченных научно-исследовательских работах, информационные материалы на отечественные, зарубежные системы-аналоги и др.), на основании которых разрабатывалось ТЗ и которые должны быть использованы при создании системы.
Техническое Задание разработано на основе следующих документов: Учебники, учебные пособия и другие материалы:
\begin{enumerate}
    \item Автоматизация управления предприятием. Модели и методы исследования предприятия: учебное пособие для студентов вузов / Д. П. Косицын, И. М. Шабалина; М-во образования и науки Рос. Федерации, Федер. гос. бюджет. образоват. учреждение высш. образования Петрозавод. гос. ун-т. – Петрозаводск: Издательство ПетрГУ, 2016 – 56 с.
\item http://docs.cntd.ru/document/gost-34-201-89 - сайт
\end{enumerate}

Нормативные правовые акты:
\begin{enumerate}
    \item Федеральный закон «Об информации, информационных технологиях и о защите информации» от 27.07.2006 N 149-ФЗ.
\item Государственные стандарты:
\item ГОСТ 34.602.2020 «Техническое задание на создание автоматизированной системы»;
\item ГОСТ РД 50-34.698-90 «Автоматизированные системы. Требования к содержанию документов»;
\item ГОСТ 19.301-79 «Единая система программной документации (ЕСПД). Программа и методика испытаний. Требования к содержанию и оформлению (с Изменениями N 1, 2)»
\end{enumerate}






\end{document}
