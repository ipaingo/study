\newcommand*{\No}{\textnumero}
\documentclass[russian, utf8, 12pt,pointsubsection,floatsubsection]{eskdtext}
\usepackage[russian]{babel} 

	\ESKDclassCode{ТП}
	\ESKDtitle{Автоматизированная система подачи заявления о выдаче водительских прав}
	\ESKDdocName{Руководства пользователя}
	\ESKDsignature{65698922.425120.022.И3.01.01}
	\ESKDcolumnII{65698922.425120.022.И3.01.01}
	\ESKDcolumnI{Программная документация. ООО «Название». РП}

	\ESKDgroup{ООО <<Название>>}
	%\ESKDauthor{Фамилия И.О.}
	\ESKDtitleAgreedBy{Директор ООО <<Название>>}{Фамилия И.О.}
	\ESKDtitleDesignedBy{Зам. директора <<Название>>}{Фамилия И.О.}

	\ESKDtitleApprovedBy{Директор ООО «Название»}{Фамилия И.О.} 
	\ESKDtitleAgreedBy{Главный технолог ООО «Название»}{Фамилия И.О.} 


 	\ESKDdate{2022/11/30} 

% \title{Text}
% \author{ }
% \date{September 2021}

\begin{document}

	% титульный лист 
	\maketitle 

	% оглавление 
	\scriptsize
	\setcounter{tocdepth}{4}
	\tableofcontents
	\normalsize
	\newpage
	
% 3.4.1. Документ содержит разделы:
% 1) введение;
% 2) назначение и условия применения;
% 3) подготовка к работе;
% 4) описание операций;
% 5) аварийные ситуации;
% 6) рекомендации по освоению.


\section{Введение}
% 3.4.2. В разделе "Введение" указывают:
% 1) область применения;
% 2) краткое описание возможностей;
% 3) уровень подготовки пользователя;
% 4) перечень эксплуатационной документации, с которыми необходимо ознакомиться
% пользователю.

\subsection{Область применения}
Пользовательский интерфейс АС для подачи и обработки заявлений в подразделении ГИБДД обеспечивает информационную поддержку деятельности оператора подразделения при выполнении задач по принятию заявлений от клиентов и дальнейшей их обработке, а также со стороны клиента система позволяет подать заявление на получение прав без посещения подразделения. Все это выполняется при помощи Desktop'ного приложения.

\subsection{Краткое описание возможностей}
Автоматизированная система обеспечивает выполнение следующих функций:
\begin{enumerate}
        \item Получение данных клиента оператором для заполнения заявки
        \item Отправка клиентом заявления на выдачу прав в подразделение

        \item Получение оператором заявления от клиента
        \item Оформление заявления клиента оператором
        \item Подтверждение оператора о закрытии заявки 

        \item Отслеживание статуса заявления клиентом

\end{enumerate}

\subsection{Уровень подготовки пользователя}
Для эффективного использования АС "ГлавПрав" должны быть определены следующие роли:
\begin{itemize}
    \item Администратор
    \item Оператор
    \item Клиент 
\end{itemize} \\

Администратор должен:
\begin{itemize}
\item Установка, настройка и мониторинг работоспособности системного программного
обеспечения
\item Установка, настройка и мониторинг прикладного программного обеспечения
\item Ведение учетных записей пользователей системы
\item Установка, модернизация, настройка параметров программного обеспечения СУБД
\item Оптимизация прикладных баз данных по времени отклика, скорости доступа к данным
\item Разработка, управление и реализация эффективной политики доступа к информа-
ции, хранящейся в прикладных базах данных
\end{itemize}
\hfill \break

Оператор должен:
\begin{itemize}
\item Иметь общие сведения о системе и ее назначении;
\item Владеть информацией об особенностях АС в объеме эксплуатационной документации;
\item Владеть информацией о работе в интерфейсе АС;
\item Добавлять/изменять/удалять записи 
\item Формировать документы для получения прав клиентом
\item Изменение статуса заявления и выбор даты посещения клиентом подразделения

\end{itemize}
\hfill \break


Клиент должен в течении непродолжительного времени разобраться в интерфейсе приложения и без обращений в колл-центр(оператору) суметь самостоятельно заполнить свои данные и сформировать заявление для отправки в подразделение.
\newline

\subsection{Перечень эксплуатационной документации, с которыми необходимо ознакомиться
пользователю}
\begin{enumerate}
    \item Инструкция по установке АС
    \item Руководство администратора АС
    \item Руководство по техническому обслуживанию АС
    \item Руководство пользователя АС.
\end{enumerate}


\section{Назначение и условия применения}
% 3.4.3. В разделе "Назначение и условия применения" указывают:
% 1) виды деятельности, функции, для автоматизации которых предназначено данное
% средство автоматизации;
% 2) условия, при соблюдении (выполнении, наступлении) которых обеспечивается
% применение средства автоматизации в соответствии с назначением (например, вид ЭВМ и
% конфигурация технических средств, операционная среда и общесистемные программные
% средства, входная информация, носители данных, база данных, требования к подготовке
% специалистов и т. п.).

\subsection{Виды деятельности, для автоматизации которых предназначено данное
 средство автоматизации}
 Автоматизированная система "ГлавПрав" предназначена для осуществления автоматизации системы подачи заявлений на получение прав, отслеживания статуса заявления и формирования документов для получения прав.

Автоматизированная ситема позволяет:
\begin{enumerate}
    \item формирование заявок и отправка их от клиента оператору подразделения
    \item обрабатывать полученнные заявления и оформлять документы для получения прав
    \item изменять статус заявления оператором
    \item устанавливать дату посещения клиентом подразделения и уведомлять его об этом
\end{enumerate}

\subsection{Условия применения}
АС "ГлавПрав" может выполнять заданные функции при соблюдении требований предъявляемых к техническому, системному и прикладному программному обеспечению.

\section{Подготовка к работе}
% 3.4.4. В разделе "Подготовка к работе" указывают:
% 1) состав и содержание дистрибутивного носителя данных;
% 2) порядок загрузки данных и программ;
% 3) порядок проверки работоспособности.

\subsection{Состав и содержание дистрибутива}
Состав дистрибутива приведен в документе "Автоматизированная система подачи заявления о выдаче водительских прав "ГлавПрав". Инструкция по установке АС "ГлавПрав"».

\subsection{Порядок установки и загрузки}
Состав дистрибутива приведен в документе "Автоматизированная система подачи заявления о выдаче водительских прав "ГлавПрав". Инструкция по установке АС "ГлавПрав"».

\subsection{Порядок проверки работоспособности}
Проверка работоспособности АС "ГлавПрав" осуществляется путем выполнения операций, описанных в разделе 4 настоящего документа.

\section{Описание операций}
% 3.4.5. В разделе "Описание операций" указывают:
% 1) описание всех выполняемых функций, задач, комплексов задач, процедур;
% 2) описание операций технологического процесса обработки данных, необходимых для
% выполнения функций, комплексов задач (задач), процедур.
% 3.4.6. Для каждой операции обработки данных указывают:
% 1) наименование;
% 2) условия, при соблюдении которых возможно выполнение операции;
% 3) подготовительные действия;
% 4) основные действия в требуемой последовательности;
% 5) заключительные действия;
% 6) ресурсы, расходуемые на операцию.
% В описании действий допускаются ссылки на файлы подсказок, размещенные на магнитных
% носителях.
В данном разделе приводится описание всех функций, существующих в автоматизированной системе подачи заявлений на получение прав "ГлавПрав".

\subsection{Порядок работы с функцей ''Получение данных клиента для заполнения заявки''}
Функция "Получение данных клиента для заполнения заявки" предоставляет возможность клиенту внести свои личные данные, необходимые для подачи заявления на получение прав.

Для этого необходимо выполнить следующие действия:
\begin{itemize}
    \item Нажать на кнопку "Создать заявление на выдачу прав"
    \item В появившейся форме указать все необходимые данные
    \item По окончанию заполнения всех полей нажать на кнопку "Отправить заявление в подразделение"
\end{itemize}

\subsection{Порядок работы с функцей ''Отправка заявления в подразделение''}
Функция "Отправка заявления в подразделение" предоставляет возможность клиенту отправить заполненное заявление в определенное подразделение для обработки. Доступ к функции открывается после последнего действия функции ''Получение данных клиента для заполнения заявки''.

Для этого необходимо выполнить следующие действия:
\begin{itemize}
    \item Из списка представленных подразделений выбрать наиболее подходящее клиенту
    \item В открывшейся информации о подразделении убедиться в наличии пункта "Имеет возможность принять ваше заявление"
    \item Нажать на кнопку "Отправить заявление в это подразделение"
    \item В открывшемся новом окне появиться уведомление о том, что заявление успешно отправлено
\end{itemize}

\subsection{Порядок работы с функцей ''Получение заявки от клиента''}
Функция "Получение заявки от клиента" предоставляет возможность оператору принимать данные клиентов.

Для этого необходимо выполнить следующие действия:
\begin{itemize}
    \item Нажать на кнопку "Входящие заявки"
    \item Из открывшегося списка выбрать необходимое заявление
    \item Откроется новое окно с полной информацией о заявлении
\end{itemize}

\subsection{Порядок работы с функцей ''Оформление заявки клиента''}
Функция "Оформление заявки клиента" предоставляет возможность оператору оформлять заявление клиента на получение прав. Доступ к функции открывается после последнего действия функции ''Получение заявки от клиента''.

Для этого необходимо выполнить следующие действия:
\begin{itemize}
    \item В окне с информацией о заявлении проверка заявление на корректность
    \item Нажать на кнопку ""сформировать документ" для дальнейшего формирования документа о получении прав
    \item В появившемся окне появиться уведомление об успешном формировании документа на получение прав
\end{itemize}

\subsection{Порядок работы с функцей ''Подтверждение закрытия заявки''}
Функция "Подтверждение закрытия заявки" предоставлять возможность сотруднику создавать приглашения для клиентов на получение прав. Доступ к функции открывается после последнего действия функции ''Оформление заявки клиента''.

Для этого необходимо выполнить следующие действия:
\begin{itemize}
    \item В окне с информацией об успешном формировании документа нажать на кнопку "Создать приглашение"
    \item В открывшемся календаре выбрать свободную дату и время, в которое клиент подойдет в подразделение
    \item Нажать на кнопку "Отправить приглашение" для уведомления клиента о выбранной дате получения прав
\end{itemize}

После всех вышеперечисленных действий обработанное заявление исчезнет из списка входящий заявок.

\subsection{Порядок работы с функцей ''Отслеживание статуса заявления''}
Функция "Отслеживание статуса заявления" предоставляет клиенту информацию о текущем состоянии его заявления. После успешного выполнения функции ''Отправка заявления в подразделение'' на главном экране пользователя появиться раздел "Моё заявление", в котором будет указываться текущий статус заявления.

Для этого необходимо выполнить следующие действия:
\begin{itemize}
    \item В разделе "Моё заявление" можно будет проверять текущий статус заявления
    \item В качестве текущего статуса будет предоставляться информации о стадии рассмотрения заявки
    \item По окончанию оформления заявления статус изменится на "Принято"
    \item После изменения статуса на "Принято" в этом же разделе юудет представлена информация о дате и времени личного визита в подразделения.
\end{itemize}




\section{Аварийные ситуации}
% 3.4.7. В разделе "Аварийные ситуации" указывают:
% 1) действия в случае несоблюдения условий выполнения технологического процесса, в
% том числе при длительных отказах технических средств;
% 2) действия по восстановлению программ и/или данных при отказе магнитных носителей
% или обнаружении ошибок в данных;
% 3) действия в случаях обнаружении несанкционированного вмешательства в данные;
% 4) действия в других аварийных ситуациях.


\subsection{Действия в случае несоблюдения условий выполнения технологического процесса}
В случае невозможности АС продолжить выполнение команд пользователей, появляются сообщения в всплывающих окнах с описанием ошибки, после чего АС возвращается в рабочее состояние, предшествовавшее неверной (недопустимой) команде или некорректному вводу данных.

Если в процессе работы АС перестает реагировать на действия пользователей, то следует перезапустить приложение. Если ошибка не устраняется, то следует обратиться к администратору АС.

\subsection{действия по восстановлению программ и/или данных при отказе}

В случае обнаружения ошибок в данных в АС следует обратиться к администратору АС. При этом необходимо указать перечень данных, содержащих ошибки, и правильные значения искаженных параметров. При нарушении работы с данными, созданными (измененными) до текущего дня, восстановление происходит из резервной копии БД. При нарушении работы с данными, созданными или отредактированными в течение текущего дня, пользователи заново вводят эти данные.

\subsection{Действия в случаях обнаружении несанкционированного вмешательства в данные}
При обнаружении несанкционированного вмешательства в данные АС необходимо обратится к администратору АС.При этом необходимо описать признаки и предполагаемый характер вмешательства,указать перечень данных, подвергшихся вмешательству и быть готовым по требованию администратора АС описать признаки аварийной ситуации и действия, которые были
выполнены пользователем непосредственно перед возникновением аварийной ситуации.

\subsection{Действия в других аварийных ситуациях}

В случае возникновения других аварийных ситуаций при работе с АС следует обратиться к администратору. При этом необходимо быть готовым по просьбе администратора описать признаки аварийной ситуации и действия, которые были выполнены непосредственно перед возникновением аварийной ситуации. 

\section{Рекомендации по освоению}

% 3.4.8. В разделе "Рекомендации по освоению" указывают рекомендации по освоению и
% эксплуатации, включая описание контрольного примера, правила его запуска и выполнения.

Все рекомендации по освоению работы в АС  "ГлавПрав" описаны в справочных материалах АС. Других специальных рекомендаций по освоению, помимо дополнительного "Руководство пользователя", не предусматривается.

\end{document}
