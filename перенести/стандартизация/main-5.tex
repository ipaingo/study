\newcommand*{\No}{\textnumero}
\documentclass[russian, utf8, 12pt,pointsubsection,floatsubsection]{eskdtext}
\usepackage[russian]{babel} 

	\ESKDclassCode{ПД}
	\ESKDtitle{Автоматизированная система подачи заявления о выдаче водительских прав "ГлавПрав"}
	\ESKDdocName{Общее описание системы}
	\ESKDsignature{65698922.425120.022.П2.01.01}
	\ESKDcolumnII{65698922.425120.022.П2.01.01}
	\ESKDcolumnI{Программная документация. ООО «Название». ОО. Ред.1}

	\ESKDgroup{ООО <<Название>>}
	%\ESKDauthor{Фамилия И.О.}
	\ESKDtitleAgreedBy{Директор ООО <<Название>>}{Фамилия И.О.}
	\ESKDtitleDesignedBy{Зам. директора <<Название>>}{Фамилия И.О.}

	\ESKDtitleApprovedBy{Директор ООО «Название»}{Фамилия И.О.} 
	\ESKDtitleAgreedBy{Главный технолог ООО «Название»}{Фамилия И.О.} 


 	\ESKDdate{2022/11/30} 

% \title{Text}
% \author{ }
% \date{September 2021}

\begin{document}

	% титульный лист 
	\maketitle 

	% оглавление 
	\scriptsize
	\setcounter{tocdepth}{4}
	\tableofcontents
	\normalsize
	\newpage
	
\section{Назначение системы}


% В разделе "Назначение системы" указывают:
% 1) вид деятельности, для автоматизации которой предназначена система;
% 2) перечень объектов автоматизации, на которых используется система;
% 3) перечень функций, реализуемых системой.

\subsection{Вид деятельности, для автоматизации которой предназначена система}
Автоматизированная система "ГлавПрав"предназначена для автоматизации управления оформлением и приёмом заявок на получение прав.
\subsection{Перечень объектов автоматизации, на которых используется система}

Объектом автоматизации является процесс подачи заявления на получение прав.

Автоматизированная система используется для:
\begin{itemize}
    \item Упрощения системы оформления заявок
    \item Перевода документооборота, ведущегося в бумажном виде, в электронный вид, упрощение и ускорение процесса оформления
    \item Повышения качества принятия управленческих решений за счёт оперативности представления и удобства отображения информации
    \item Уменьшения времени затрачиваемого на подачу и оформление заявки со стороны клиента, удобного предоставления требуемых документов
\end{itemize}

\subsection{Перечень функций, реализуемых системой}

Автоматизированная система подразделена на следующие подсистемы:
\begin{enumerate}
    \item Подсистема формирования заявления
    \item Подсистема обработки заявления
    \item Подсистема отслеживания заявлений
\end{enumerate}

Каждая выделенная подсистема обладает набором реализуемых функций:
\begin{enumerate}
    \item Подсистема формирования заявления имеет следующие функции:
    \begin{itemize}
        \item Получение данных клиента для заполнения заявки
        \item Отправка заявления в подразделение
    \end{itemize}
    \item Подсистема обработки заявления имеет следующие функции:
    \begin{itemize}
        \item Получение заявки от клиента
        \item Оформление заявки клиента
        \item Подтверждение закрытия заявки
    \end{itemize}
    \item Подсистема отслеживания заявлений имеет следующие функции:
    \begin{itemize}
        \item Отслеживание статуса заявления
    \end{itemize}
\end{enumerate}


\section{Описание системы}
% 2.11.3. В разделе "Описание системы" указывают:
% 1) структуру системы и назначение ее частей;
% 2) сведения об АС в целом и ее частях, необходимые для обеспечения эксплуатации
% системы;
% 3) описание функционирования системы и ее частей.

\subsection{Структура системы и назначение ее частей}
Информационная система состоит из следующих модулей:
\begin{itemize}
    \item Подсистема клиента - подсистема реализует необходимый функционал для создания и заполнения заявки на получение прав, а также отслеживание её статуса;
    \item Подсистема подразделения - подсистема реализует необходимый функционал для обработки получаемых от клиентов заявок;
    \item Подсистема коммуникации - подсистема реализует необходимый функционал для взаимодействия клиента с подразделением;
\end{itemize}

\subsection{Cведения об АС в целом и ее частях, необходимые для обеспечения эксплуатации системы}
Разрабатываемая АС ориентирована на выполнение целевых функций системы, обеспечивающее корректное выполнение всех поставленных перед системой задач. Для функционирования системы необходимо:
\begin{itemize}
\item База данных (СУБД) PostgreSQL или MySQL;
\item Apache HTTP Server версии 2.2.21 (или выше);
\item CMS Drupal 7
\item PHP версии 5.2 (или выше);
\item Система управления контентом с открытым кодом.
\end{itemize}

Требования к ПК для взаимодействия с клиентской частью:
\begin{itemize}
    \item intel pentium;
    \item 4gb оперативной памяти;
    \item 80gb объем жесткого диска;
    \item монитор;
    \item компьютерная мышь;
    \item клавиатура.
\end{itemize}


\subsection{Описание функционирования системы и ее частей}
\begin{itemize}
    \item Пользователь отправляет запрос через веб-браузер на веб-сервис
    \item Веб-сервис получает запрос пользователя и отправляет его на обработку серверу;
    \item Сервер делает запрос в базу данных;
    \item Сервер формирует разметку страницы для пользователя и отдает ее веб-серверу;
    \item Пользователь получает ответ от веб-сервера.
\end{itemize}

\section{Описание взаимосвязей АС с другими системами}

% 2.11.4. В разделе "Описание взаимосвязей АС с другими системами" указывают:
% 1) перечень систем, с которыми связана данная АС;
% 2) описание связей между системами;
% 3) описание регламента связей;
% 4) описание взаимосвязей АС с подразделениями объекта автоматизации.
Внешние системы не используются.

\section{Описание подсистем}
% 2.11.5. В разделе "Описание подсистем" указывают:
% 1) структуру подсистем и назначение ее частей;
% 2) сведения об подсистемах и их частях, необходимые для обеспечения их
% функционирования;
% 3) описание функционирования подсистем и их частей.

\subsection{Структура подсистем и назначение ее частей}
Информационно-технологические компоненты, включают в себя
следующие подсистемы:
\begin{itemize}
    \item Подсистема клиента - подсистема реализует необходимый функционал для создания и заполнения заявки на получение прав, а также отслеживание её статуса;
    \item Подсистема подразделения - подсистема реализует необходимый функционал для обработки получаемых от клиентов заявок;
    \item Подсистема коммуникации - подсистема реализует необходимый функционал для взаимодействия клиента с подразделением;
\end{itemize}

\subsection{Сведения об подсистемах и их частях, необходимые для обеспечения их функционирования}

Взаимодействие с подсистемами автоматизированной системы происходит через веб-браузер. Для опеспечения функционирования системы необходим доступ в интернет.

\subsection{Описание функционирования подсистем и их частей}
\subsubsection{Описание функции ''Получение данных клиента для заполнения заявки''}
Функция "Получение данных клиента для заполнения заявки" позволяет:
\begin{itemize}
    \item Получать необходимые документы
    \item Проверять наличие всех требуемых данных
    \item Формировать заявление для отправки в подразделение
\end{itemize}

\subsubsection{Описание функции ''Отправка заявления в подразделение''}
Функция "Отправка заявления в подразделение" позволяет:
\begin{itemize}
    \item Получать информацию о выборе подразделения
    \item Проверять возможность подразделения принять заявление
    \item Отправлять сформированное заявление в подразделение
\end{itemize}

\subsubsection{Описание функции ''Получение заявки от клиента''}
Функция "Получение заявки от клиента" позволяет:
\begin{itemize}
    \item Принимать данные со стороны клиента 
    \item Добавлять заявления в список входящих заявок
    \item Предоставлять заявления сотруднику
\end{itemize}


\subsubsection{Описание функции ''Оформление заявки клиента''}
Функция "Оформление заявки клиента" позволяет:
\begin{itemize}
    \item Проверять заявления на корректность
    \item Формировать документ на получение прав
    \item Передавать сформированный документ на закрытие
\end{itemize}

\subsubsection{Описание функции ''Подтверждение закрытия заявки''}
Функция "Подтверждение закрытия заявки" позволяет:
\begin{itemize}
    \item Предоставлять сотруднику документ о получении прав
    \item Назначать дату приглашения клиента на получение прав
    \item Переводить документ из списка текущих заявок
\end{itemize}

\subsubsection{Описание функции ''Отслеживание статуса заявления''}
Функция "Отслеживание статуса заявления" позволяет:
\begin{itemize}
    \item Предоставлять информацию о стадии рассмотрения заявки
    \item Уведомлять клиента о завершении обработки заявления
    \item Предоставлять дату личного визита в подразделения по окончанию оформления заявления
\end{itemize}


\end{document}
