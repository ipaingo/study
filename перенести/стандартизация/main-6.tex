\newcommand*{\No}{\textnumero}
\documentclass[russian, utf8, 12pt,pointsubsection,floatsubsection]{eskdtext}
\usepackage[russian]{babel} 

	\ESKDclassCode{ТП}
	\ESKDtitle{Автоматизированная система регистрации заявок "Auto Tire" для шиномонтажа}
	\ESKDdocName{Паспорт}
	\ESKDsignature{65698922.425120.022.И3.01.01}
	\ESKDcolumnII{65698922.425120.022.И3.01.01}
	\ESKDcolumnI{Программная документация.«Auto Tire». П2. Ред.1}


	\ESKDgroup{ООО <<Центр Шиномонтажа>>}
	\ESKDtitleAgreedBy{Директор ООО «Центр Шиномонтажа»}{Косицын Д.П.}
	\ESKDtitleDesignedBy{Зам. директора ООО «Центр Шиномонтажа»}{Андреев А.А.}

	\ESKDtitleApprovedBy{Директор ООО «Центр Шиномонтажа»}{Васильев В.В.} 
	\ESKDtitleAgreedBy{Главный технолог ООО «Центр Шиномонтажа»}{Богданов Б.Б.} 
 	\ESKDdate{2022/07/11} 


% \title{Text}
% \author{ }
% \date{September 2021}

\begin{document}

	% титульный лист 
	\maketitle 

	% оглавление 
	\scriptsize
	\setcounter{tocdepth}{4}
	\tableofcontents
	\normalsize
	\newpage
	
% Документ содержит разделы: 
% 1) общие сведения об АС; 
% 2) основные характеристики АС; 
% 3) комплектность; 
% 4) свидетельство (акт) о приемке; 
% 5) гарантии изготовителя (поставщика); 
% 6) сведения о рекламациях. 
% 1 СОДЕРЖАНИЕ:
% 1 СОДЕРЖАНИЕ:	3
% 2 ОБЩИЕ СВЕДЕНИЯ ОБ АВТОМАТИЗИРОВАННОЙ СИСТЕМЕ	5
% 3 ОСНОВНЫЕ ХАРАКТЕРИСТИКИ АС	6
% 3.1 Состав функций	6
% 3.2 Принцип функционирования АС	6
% 3.3 Регламент и режимы функционирования АС	6
% 3.4 Совместимость АС с другими системами	6
% 4 КОМПЛЕКТНОСТЬ	7
% 5 СВИДЕТЕЛЬСТВО О ПРИЕМКЕ	8
% 6 ГАРАНТИИ ИЗГОТОВИТЕЛЯ	9
% 7 СВЕДЕНИЯ О РЕКЛАМАЦИЯХ	10
 
\section{ОБЩИЕ СВЕДЕНИЯ ОБ АВТОМАТИЗИРОВАННОЙ СИСТЕМЕ} 
% В разделе "Общие сведения об АС" указывают наименование АС, ее обозначение, присвоенное разработчиком, наименование предприятия-поставщика и другие сведения об АС в целом. 

Система Auto Tire предназначена для автоматизации процессов регистрации заявок на
посещение косметического салона
\subsection{Наименование проектируемой автоматизируемой системы}
Полное наименование системы: Автоматизированная система регистрации заявок "Auto Tire" для шиномонтажа.\\

Краткое наименование системы: АС "Auto Tire".\\

Номер контракта: №1/11-11-11-001 от 11.11.2022.\\
\subsection{Организации, участвующие в разработке}
\subsubsection{Заказчик}
Заказчиком системы является коммерческое предприятие "Auto Tire"(ООО "Центр Шиномонтажа").\\
Контактные данные директора Косицына Д.П.: 
\begin{itemize}
    \item телефон: +7(911)000-00-00 
    \item электронная почта: mail@gmail.com
\end{itemize}

Адрес заказчика: 185022 г. Петрозаводск,ул. Шотмана, д.9.\\

Разработчиком системы является ООО "Разраб".\\

Контактные данные директора Ленина В.И.: 
\begin{itemize}
    \item телефон: +7(911)000-11-22
    \item электронная почта: lenin@gmail.com
\end{itemize}


\section{ОСНОВНЫЕ ХАРАКТЕРИСТИКИ АС}
% В разделе "Основные характеристики АС" должны быть приведены: 
% 1) сведения о составе функций, реализуемых АС, в том числе измерительных и управляющих; 
% 2) описание принципа функционирования АС; 
% 3) общий регламент и режимы функционирования АС и сведения о возможности изменения режимов ее работы; 
% 4) сведения о совместимости АС с другими системами. 
\subsection{Состав функций}
Система выполняет следующие функции:
\begin{enumerate}
    \item Функции пользователя потребителя:
    \begin{enumerate}
        \item просмотр перечня услуг и информации по ним;
        \item просмотр актуального расписания шиномонтажа;
        \item заполнение и отправка заявки.
    \end{enumerate}
    \item Процессы для администраторов салона включают в себя:
    \begin{enumerate}
        \item мониторинг заявок на услуги шиномонтажа;
        \item составление/изменение расписания бригад;
        \item автоматическое информирование клиентов о наличии записи.
    \end{enumerate}
\end{enumerate}

\subsection{Принцип функционирования АС}



Система построена по клиент-серверной схеме взаимодействия с тонким клиентом. Веб-страницы, хранящиеся и генерируемые на серверной стороне, загружаются по сети Интернет на компьютеры конечных пользователей. \\

Программное обеспечение обеспечивает управление информационным наполнением с
помощью графического интерфейса, не требующего от пользователей знания
специализированных языков программирования.\\

\subsection{Регламент и режимы функционирования АС}

Система должна обеспечивать возможность одновременной работы 150 пользователей для подсистемы потребителя услуг (клиентов), и не менее 2 пользователей для подсистемы пользователя, который обрабатывает услуги при следующих характеристиках времени отклика системы:

– для операций навигации по экранным формам системы – не более 5 сек;

– для операций формирования справок и выписок – не более 10 сек.\\


Система функционирует в следующих режимах:
\begin{itemize}
    \item штатный режим, при котором обеспечивается выполнение всех функций системы;
    \item режим обслуживания, необходимый для проведения обслуживания, реконфигурации и
пополнения технических и программных средств системы новыми компонентами.
\end{itemize}
Основным режимом функционирования системы является штатный режим, в котором
выполняются все функции, реализованные в системе.


\subsection{Совместимость АС с другими системами}
АС "Auto Tire" не осуществляет взаимодействие с другими системами.

\section{КОМПЛЕКТНОСТЬ} 
% В разделе "Комплектность" указывают все непосредственно входящие в состав АС комплексы технических и программных средств, отдельные средства, в том числе носители данных и эксплуатационные документы. 
В состав Системы входят следующие комплексы технических и программных средств,
отдельные средства, в том числе носители данных и эксплуатационные документы:
\begin{enumerate}
    \item Документ «Описание предметной области» согласно со стандартом оформления документации ГОСТ 2.105-95;
    \item Документ «Техническое задание» согласно ГОСТ 34.602;
    \item Документ «Общее описание» согласно ГОСТ Р59795-2021;
    \item Документ «Руководство пользователя» согласно РД 50-34.698-90 п.3.4.;
    \item Документ «Программа и методики испытаний» согласно ГОСТ 19.301-79;
    \item Документа «Паспорт» согласно ГОСТ 50-34.698-90 п.2.8.;
\end{enumerate}

Все созданные в рамках настоящей работы программные изделия (за исключением покупных) передаются Заказчику, как в виде готовых модулей, так и в виде исходных кодов, представляемых в электронной форме на стандартном машинном носителе.\\

\section{СВИДЕТЕЛЬСТВО О ПРИЕМКЕ}
% В разделе "Свидетельство о приемке" приводят дату подписания акта о приемке АС в промышленную эксплуатацию и фамилии лиц, подписавших акт.
На момент разработки документа свидетельство (акт) о приемке не составлялось.


\section{ГАРАНТИИ ИЗГОТОВИТЕЛЯ} 
% В разделе "Гарантии изготовителя" приводят сроки гарантии АС в целом и ее отдельных составных частей, если эти сроки не совпадают со сроками гарантии АС в целом.
Гарантия изготовителя включает устранение неисправностей в работе АС "Auto Tire", связанных
с ошибками (под ошибкой понимается устойчиво воспроизводимое несоответствие фактического
поведения системы требованиям, определенным Техническим заданием) в программном
обеспечении. В случае выявления Исполнителем неисправностей, возникших не по вине
Исполнителя, инцидент может быть разрешен по согласованию сторон, в том числе может быть
назначена независимая экспертиза системы с целью определения ответственности сторон.\\

Информирование Исполнителя по обнаруженным ошибкам осуществляется посредством
официального уведомления письмом/телеграммой/другим видом официального сообщения, а
также направления копии уведомления на официальный адрес электронной почты Исполнителя.

\section{СВЕДЕНИЯ О РЕКЛАМАЦИЯХ}
% В разделе "Сведения о рекламациях" регистрируют все предъявленные рекламации, их краткое содержание и меры, принятые по рекламациям.
На момент разработки документа в адрес Исполнителя рекламации не поступали.


\end{document}