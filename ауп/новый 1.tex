\newcommand*{\No}{\textnumero}
\documentclass[russian, utf8, 12pt,pointsubsection,floatsubsection]{eskdtext}
\usepackage[utf8]{inputenc}
\usepackage[russian]{babel} 

\ESKDclassCode{ТП}
\ESKDtitle{Система учета процессов компании по установке автономных канализаций}
\ESKDdocName{Техническое задание}
\ESKDsignature{65698922.425120.022.ТЗ.01.01}
\ESKDcolumnII{65698922.425120.022.ТЗ.01.01}
\ESKDcolumnI{Программная документация. ООО «Название». ТЗ. Ред.1}

\ESKDgroup{ООО «Название»}
\ESKDtitleAgreedBy{Директор ООО «Название»}{Фамилия И.О.}
\ESKDtitleDesignedBy{Зам. директора ООО «Название»}{Фамилия И.О.}

\ESKDtitleApprovedBy{Директор ООО «Название»}{Фамилия И.О.} 
\ESKDtitleAgreedBy{Главный технолог ООО «Название»}{Фамилия И.О.} 


\ESKDdate{2024/10/26} 

\begin{document}

\maketitle 

\scriptsize
\setcounter{tocdepth}{4}
\tableofcontents
\normalsize
\newpage


\section{Общие сведения}

\subsection{Полное наименование системы и ее условное обозначение}
Полное наименование: Автоматизированная информационная система "Мастер CRM".\\

Условное обозначение: "Мастер CRM"

\subsection{Номер договора (контракта)}
Номер договора: \No ПО-24101 от 26.10.2024

\subsection{Наименования организации-Заказчика и организаций-участников работ}
Компания-Заказчик: ООО "Название"\\

Адрес Заказчика: 185005, Республика Карелия, г. Петрозаводск, ул. Улица, д. 1\\
Телефон/факс: +7 (8142) 11-11-11\\


Разработчики: студентка института математики и информационных технологий ПетрГУ, специальности «Информационные системы и технологии», группы 22405 Зименкова С.Э.\\

Адрес разработчика: г. Петрозаводск, пр-т Ленина, д. 33

\subsection{Перечень документов, на основании которых создается система}
Работа выполняется на основании договора №241026-ПО от 26.10.2024 между Заказчиком ООО "Название" и разработчиками ПетрГУ.

\subsection{Плановые сроки начала и окончания работы по созданию системы}
Плановый срок начала работ по созданию системы --- 1 декабря 2024 года.
Плановый срок окончания работ по созданию системы --- 30 мая 2025 года.

\subsection{Источники и порядок финансирования работ}
Источником финансирования является оплата по факту выполнения работ, осуществляемая Заказчиком.
Порядок финансирования определяется условиями Договора №241026-ПО от 26.10.2024

\subsection{Порядок оформления и предъявления Заказчику результатов работ по созданию системы}
Разработчики по завершении работ по созданию системы предоставляет программу: автоматизированная информационная система «Мастер CRM», осуществляет ее установку на оборудование Заказчика и предоставляет необходимую документацию.\\

Порядок оформления и предъявления результатов работ должен соответствовать требованиям комплекса стандартов и руководящих документов на автоматизированные системы: ГОСТ 34.201-89, ГОСТ 34.602-2020, ГОСТ 34.601-90.
Разработка проектных решений Системы, ее подсистем и (или) ее частей должно осуществляться в соответствии с данным Техническим заданием и исходными данными, предоставляемыми Заказчиком. В случае необходимости представители Исполнителя проводят обследование объекта автоматизации с участием представителей Заказчика.


\subsection{Перечень нормативно-технических документов, методических материалов, использованных при разработке ТЗ}
При разработке Системы и создании проектно-эксплуатационной документации Исполнитель должен руководствоваться следующими нормативными документами:
\begin{enumerate}
Договор №ПО-24101 от 26.10.2024
ГОСТ 2.105-95. Единая система конструкторской документации. Общие требования к текстовым документам
ГОСТ 19.201-78. ТЕХНИЧЕСКОЕ ЗАДАНИЕ. ТРЕБОВАНИЯ К СОДЕРЖАНИЮ И ОФОРМЛЕНИЮ;
ГОСТ 34.601-90. Комплекс стандартов на автоматизированные системы. Автоматизированные системы. Стадии создания;
ГОСТ 34.602.2020 Техническое задание на создание автоматизированной системы.
\end{enumerate}

\subsection{Определения, обозначения и сокращения}
АИС --- Автоматизированная Информационная Система\\
АСУ --- Автоматизированная Система Управления\\
ОС --- операционная система

\section{Назначение и цели создания системы}

\subsection{Назначение системы}
Автоматизированная информационная система "Мастер CRM" предназначена для управления процессами компании по установке автономных канализаций.

\subsection{Цели создания системы}
Основными целями внедрения системы являются:
itemize
Оптимизация процесса взаимодействия менеджера с клиентом;
Упрощение оперативной регистрации и отслеживания этапа работ по заказу клиента;
Упрощение контроля остатков товаров на складе;
Перевод документооборота, ведущегося в виде разрозненных файлов в разных системах, в единую информационную систему;
Повышение качества принятия управленческих решений за счет оперативности представления и удобства отображения информации;
Предоставление сотрудникам компании доступа к системе с любого устройства, имеющего доступ в Интернет, вне зависимости от расположения.
itemize

\section{Характеристика объекта автоматизации}

Характеристика объекта автоматизации изложена в документе описания предметной области «Система учета процессов компании по установке автономных канализаций» и в документе «Обследование компании», разработанных в процессе обследования предприятия Заказчика.

\section{Требования к системе}
\subsection{Требования к системе в целом}
\point{Требования к структуре и функционированию системы}
\subpoint{Перечень подсистем, их назначение и основные характеристики}

В состав автоматизированной системы "Мастер CRM" должны входить следующие подсистемы:
itemize
Подсистема регистрации звонков
Подсистема задач по заказам клиентов
Подсистема контроля остатков на складе
Подсистема хранения информации о клиенте
itemize

Подсистема регистрации звонков предназначена для упрощения процесса записи звонка клиента в компанию. В этой подсистеме менеджер отмечает основную информацию о клиенте и содержание звонка, чтобы после продолжить работу по задаче.
Подсистема задач по заказам клиентов предназначена для отслеживания текущего этапа работ по заказу каждого отдельного клиента. В этой подсистеме сотрудники, выполняющие работы по заказу, отмечают выполненные работы и плановую активность, чтобы упростить отслеживание задач.
Подсистема контроля остатков на складе предназначена для упрощения отслеживания поступления и отгрузки товаров, регулярно приобретаемых клиентами.
Подсистема хранения информации о клиенте предназначена для отображения наиболее полной информации, которая может понадобиться при дальнейшей работе с каждым клиентом. Эта подсистема обеспечивает доступ сотрудников к базе данных клиентов с помощью удобного интерфейса, который позволяет просмотреть всю информацию о каждом клиенте, подписанных документах и проведенных работах.

\subpoint{Требования к способам и средствам связи для информационного обмена между компонентами системы}

Компоненты автоматизированной системы должны связываться друг с другом в обоюдном порядке для получения, обмена и обработки информации о клиентах, товарах, работах, документах и заказах. 

\point{Требования к численности и квалификации персонала системы}\\ 
Пользователи системы --- сотрудники компании --- должны иметь опыт работы с персональным компьютером на базе операционных систем Microsoft Windows и свободно осуществлять базовые операции в стандартных Windows. По возможности пользователи должны иметь опыт работы с мобильным устройством (смартфон или планшет) для осуществления удаленного доступа к системе при наличии доступа к Интернет.
\point{Показатели назначения}\\
Целевое назначение системы должно сохраняться на протяжении всего срока эксплуатации компании. Срок эксплуатации определяется сроком устойчивой работы аппаратных средств, своевременным проведением работ по замене (обновлению) аппаратных средств, по сопровождению программного обеспечения системы и его модернизации.\\
Время выполнения запросов информации в АИС определяется на стадии проектирования системы.\\
В АИС должны быть обеспечены возможности по созданию, добавлению, изменению и удалению полей данных в пользовательском интерфейсе.
Прочие показатели назначения АИС разрабатываются после проведения предварительного обследования.

\point{Требования к надежности}\\ 
Система должна сохранять работоспособность и обеспечивать восстановление своих функций при возникновении следующих внештатных ситуаций:
itemize
при сбоях в системе электроснабжения аппаратной части, приводящих к перезагрузке ОС, восстановление программы должно происходить после перезапуска ОС и запуска исполняемого файла системы;
при ошибках в работе аппаратных средств (кроме носителей данных и программ) восстановление функции системы возлагается на ОС;
при ошибках, связанных с программным обеспечением (ОС и драйверы устройств), восстановление работоспособности возлагается на ОС.
itemize

Для защиты аппаратуры от бросков напряжения и коммутационных помех должны применяться сетевые фильтры и источники бесперебойного питания.\\
Время восстановления работоспособности прикладного ПО АИС при любых сбоях и отказах не должно превышать одного рабочего дня.\\
Уровень надежности должен достигаться согласованным применением организационных, организационно-технических мероприятий и программно-аппаратных средств.\\

Надежность должна обеспечиваться за счет:
itemize
Применения технических средств, системного и базового программного обеспечения, соответствующих классу решаемых задач;
Разграничения прав доступа к системе;
Возможности восстановления данных с внешнего накопителя после восстановления активного накопителя;
Своевременного выполнения процессов администрирования Системы;
Соблюдения правил эксплуатации и технического обслуживания программно-аппаратных средств.
itemize


\point{Требования к безопасности}\\

Пользователи системы при работе с компьютером или другим вычислительным устройством, соединенным с сетью, должны соблюдать технику безопасности. Все внешние элементы технических средств системы, находящиеся под напряжением, должны иметь защиту от случайного прикосновения, а сами технические средства иметь зануление или защитное заземление в соответствии с ГОСТ 12.1.030-81 и Правилам устройства электроустановок.\\

Общие требования пожарной безопасности должны соответствовать нормам на бытовое электрооборудование. В случае возгорания не должно выделяться ядовитых газов и дымов. После снятия электропитания должно быть допустимо применение любых средств пожаротушения.

\point{Требования к эргономике и технической эстетике}\\
Работа пользователя с системой должна осуществляться посредством графического интерфейса. Графический интерфейс должен обладать следующими свойствами:
itemize
соответствие современным эргономическим требованиям и обеспечение удобного доступа к основным функциям и операциям системы;
управление с помощью набора экранных меню, кнопок, значков и других элементов;
простота и удобство использования;
доступ посредством устройства с доступом к Интернету;
все надписи экранных форм, а также сообщения, выдаваемые
пользователю (кроме системных сообщений) должны быть на русском языке;
должен быть выполнен в едином графическом дизайне, приятном для восприятия и повторяющем дизайн веб-сайта компании.
itemize

\point{Требования к эксплуатации, техническому обслуживанию, ремонту и хранению компонентов системы}\\
Система должна обеспечивать круглосуточный режим работы.

Условия эксплуатации и периодичность обслуживания технических средств системы должны соответствовать требованиям по эксплуатации, техническому обслуживанию, ремонту и хранению, изложенным в документации завода-изготовителя (производителя).\\

Периодическое техническое обслуживание используемых технических средств должно проводиться в соответствии с требованиями технической документации изготовителей, но не реже одного раза в год.\\

Для электропитания технических средств должна быть предусмотрена трехфазная четырехпроводная сеть с глухо заземленной нейтралью 380/220 В (+10-15)\% частотой 50 Гц (+1-1) Гц. Каждое техническое средство запитывается однофазным напряжением 220 В частотой 50 Гц через сетевые розетки с заземляющим контактом.\\

Для обеспечения выполнения требований по надежности должен быть создан комплект запасных изделий и приборов (ЗИП).\\

Размещение оборудования, технических средств должно соответствовать требованиям техники безопасности, санитарным нормам и требованиям пожарной безопасности. Все пользователи системы должны соблюдать правила эксплуатации электронной вычислительной техники.

\point{Требования к защите информации от несанкционированного доступа}

Размещение помещений и их оборудование должны исключать возможность бесконтрольного проникновения в них посторонних лиц и обеспечивать сохранность находящихся в этих помещениях конфиденциальных документов и технических средств.

Компоненты подсистемы защиты должны обеспечивать:
\begin{enumerate}
идентификацию пользователя;
проверку полномочий пользователя при работе с системой;
разграничение доступа пользователей на уровне задач и информационных массивов.
\end{enumerate}

АИС должна использовать "слепые" пароли (при наборе пароля его символы не показываются на экране или заменяются одним типом символов). АИС должна автоматически блокировать сессии пользователей и приложений по заранее заданным временам отсутствия активности со стороны пользователей и приложений. АИС должна использовать многоуровневую систему защиты.\\

В системе должны быть предусмотрены механизмы исправления неверно проведенных операций. При этом должна соблюдаться принятая Заказчиком технология, предусматривающая подобные случаи, а также обеспечиваться регистрация исправительных действий в соответствующих журналах для последующего контроля.

\point{Требования по сохранности информации при авариях}
Используемые аппаратные и системные платформы должны обеспечивать сохранность и целостность информации в системе при полном или частичном отключении электропитания, аварии сетей телекоммуникации, полном или частичном отказе технических средств системы.

В системе должны быть предусмотрены меры, обеспечивающие целостность данных в случае отказа аппаратных средств или программного обеспечения.

Сохранность информации в системе должна быть обеспечена при:
\begin{enumerate}
отключении электропитания;
отказе компьютера, на котором работает программа;
временном отказе линий связи.
\end{enumerate}

Программное обеспечение должно восстанавливать свое функционирование при корректном перезапуске аппаратных средств. Должна быть предусмотрена возможность организации автоматического и (или) ручного резервного копирования данных системы средствами системного и базового программного обеспечения (ОС, СУБД), входящего в состав программно-технического комплекса Заказчика.

\point{Требования к защите от влияния внешних воздействий}
Защита от влияния внешних воздействий должна обеспечиваться средствами программно-технического комплекса Заказчика.

\point{Требования к патентной частоте}
Установка системы в целом, как и установка отдельных частей системы, не должна предъявлять дополнительных требований к покупке лицензий на программное обеспечение сторонних производителей.

\point{Требования по стандартизации и унификации}
Разрабатываемая система должна соответствовать:
\begin{enumerate}
ГОСТ 34.601-90 «Комплекс стандартов на автоматизированные системы. Автоматизированные системы. Стадии создания»;
ГОСТ 34.201-89 «Информационная технология. Комплекс стандартов на автоматизированные системы. Виды, комплексность и обозначение документов при создании автоматизированных систем»;
РД 50-34.698-90 «Автоматизированные системы. Требования к содержанию документов».
ГОСТ 34.603-92 «Информационная технология. Виды испытаний автоматизированных систем».
\end{enumerate}

В системе должны использоваться (при необходимости) общероссийские классификаторы и единые классификаторы и словари для различных видов алфавитно-цифровой и текстовой информации.

\point{Дополнительные требования}

Для корректной работы АИС Заказчику потребуется компьютерная техника (сервер), подключенная к сети Интернет с помощью статического ("белого") IP-адреса.

\subsection{Требования к функциям (задачам), выполняемым системой}

В состав автоматизированной системы "Мастер CRM" должны входить следующие подсистемы:
itemize
Подсистема регистрации звонков
Подсистема задач по заказам клиентов
Подсистема контроля остатков на складе
Подсистема хранения информации о клиенте
itemize

Подсистема регистрации звонков имеет следующие функции:
itemize
Отображение информации о клиенте компании по номеру телефона
Заполнение информации о новом клиенте или контактном лице
Регистрация нового заказа клиента
itemize

Подсистема задач по заказам клиентов имеет следующие функции:
itemize
Отображение информации о клиенте и заказе
Внесение и отображение информации о текущем этапе работ и планируемых работах
Отображение информации о результатах выполненных работ
itemize

Подсистема контроля остатков на складе имеет следующие функции:
itemize
Отображение информации о каждой позиции товаров (номенклатура, стоимость, остаток на складе)
Внесение информации о поступлении и отгрузке товаров со склада
Отображение уведомления при низком остатке товара на складе и необходимости провести закупку
itemize

Подсистема хранения информации о клиенте имеет следующие функции:
itemize
Отображение всех клиентов компании
Отображение и изменение подробной информации о клиенте
Отображение и изменение информации о договорах и других документах клиента
Внесение и отображение информации о сервисном обслуживании оборудования клиента
itemize


%%%%%%%%%%%%%%%%%%%%%%%%%%%%%%%%%%%%%%%

% Отображение информации о клиенте компании по номеру телефона
% Заполнение информации о новом клиенте или контактном лице
% Регистрация нового заказа клиента

\subsubsection{Описание требований к функции ''Отображение информации о клиенте компании по номеру телефона''}
Функция "Отображение информации о клиенте компании по номеру телефона" должна по введенному пользователем номеру телефона вывести в графический интерфейс информацию о клиенте.

Для этого необходимо выполнить следующие подзадачи:
itemize
Получение номера телефона входящего звонка
Проверка наличия номера в базе данных
Отображение в интерфейсе ФИО контактного лица, ФИО клиента, адреса, проведенных для этого клиента работ, если номер есть в базе данных
Переход к функции "Заполнение информации о новом клиенте или контактном лице", если номер отсутствует в базе данных
itemize

\subsubsection{Описание требований к функции ''Заполнение информации о новом клиенте или контактном лице''}
Функция "Заполнение информации о новом клиенте или контактном лице" должна добавлять запись о новом клиенте или контактном лице клиента, если ранее номер телефона входящего звонка не был зарегистрирован в базе данных.

Для этого необходимо выполнить следующие подзадачи:
itemize
Получение номера телефона входящего звонка
Получение ФИО клиента
Получение ФИО контактного лица
Сохранение информации в базе данных
itemize

\subsubsection{Описание требований к функции ''Регистрация нового заказа клиента''}
Функция "Регистрация нового заказа клиента" должна добавлять запись о новом заказе клиента.

Для этого необходимо выполнить следующие подзадачи:
itemize
Получение ФИО и номера контактного лица, клиента
Получение информации о заказе клиента (адрес, описание заказа, дополнительная информация)
Присвоение заказу уникального номера
Сохранение информации о заказе в базе данных
itemize

%%%%%%%%%%%%%%%%%%%%%%%%%%%%%%%%%%%%%%%
% Отображение информации о клиенте и заказе
% Внесение и отображение информации о текущем этапе работ и планируемых работах
% Отображение информации о результатах выполненных работ

\subsubsection{Описание требований к функции ''Отображение информации о клиенте и заказе''}
Функция "Отображение информации о клиенте и заказе" должна выводить в графический интерфейс информацию о клиенте и заказе этого клиента.

Для этого необходимо выполнить следующие подзадачи:
itemize
Получение номера заказа клиента
Отображение в интерфейсе ФИО клиента, информации о заказе (адрес, описание заказа, дополнительная информация), текущего этапа работ, последней записи о выполненных работах
Переход к функции "Внесение и отображение информации о текущем этапе работ и планируемых работах" по нажатию на кнопку "Ввести работы по заказу"
itemize

\subsubsection{Описание требований к функции ''Внесение и отображение информации о текущем этапе работ и планируемых работах''}
Функция "Внесение и отображение информации о текущем этапе работ и планируемых работах" должна отображать историю записей о работах по заказу и позволять пользователю добавить новую запись. 

Для этого необходимо выполнить следующие подзадачи:
itemize
Отображение истории записей в графическом интерфейсе
Для каждой записи отображаются: дата, содержание работ, планируемые работы, текущий этап работ
Изменение существующих записей или создание новой записи
itemize

\subsubsection{Описание требований к функции ''Внесение и отображение информации о результатах выполненных работ''}
Функция "Отображение информации о результатах выполненных работ" должна позволять пользователю после выполнения работ по заказу отметить заказ как выполненный и указать результат работ.

Для этого необходимо выполнить следующие подзадачи:
itemize
Получение номера заказа
Отметка заказа как выполненного и выбор результата из списка "Работа окончена", "Закрывающие документы получены" и "Неуспешное завершение работ".
Получение дополнительных комментариев по результатам выполненных работ.
Отображение страницы заказа с указанными сведениями о завершении работ при открытии заказа по функции "Отображение информации о клиенте и заказе".
itemize

%%%%%%%%%%%%%%%%%%%%%%%%%%%%%%%%%%%%%%%

% Отображение информации о каждой позиции товаров (номенклатура, стоимость, остаток на складе)
% Внесение информации о поступлении и отгрузке товаров со склада
% Отображение уведомления при низком остатке товара на складе и необходимости провести закупку

\subsubsection{Описание требований к функции ''Отображение информации о каждой позиции товаров''}
Функция "Отображение информации о каждой позиции товаров" должна отображать номенклатуры, стоимость и остаток на складе по всем товарам, которые продаются компанией, в виде таблицы в графическом интерфейсе. Пользователь должен иметь возможность осуществить поиск по названию, отсортировать таблицу по алфавиту или остатку. Пользователь должен иметь возможность добавить новый товар или изменить информацию о товаре.

Для этого необходимо выполнить следующие подзадачи:
itemize
Получение строки поиска или параметра сортировки
Отображение таблицы в интерфейсе
Получение информации о товаре при добавлении или изменении
Сохранение информации о товаре в базе данных
itemize

\subsubsection{Описание требований к функции ''Внесение информации о поступлении и отгрузке товаров со склада''}
Функция "Внесение информации о поступлении и отгрузке товаров со склада" должна отображать информацию в базе данных при вносе записей о поступлении или отгрузке товаров.

Для этого необходимо выполнить следующие подзадачи:
itemize
Получение информации о поступлении товаров: номенклатура, стоимость, количество, поставщик
Получение информации об отгрузке товаров: номенклатура, стоимость, количество, покупатель
Сохранение информации в базе данных
itemize

\subsubsection{Описание требований к функции ''Отображение уведомления при низком остатке товара на складе и необходимости провести закупку''}
Функция "Отображение уведомления при низком остатке товара на складе и необходимости провести закупку" должна выводить оповещение для пользователя, если после отгрузки товара его количество на складе стало меньше контрольного.

Для этого необходимо выполнить следующие подзадачи:
itemize
Получение информации об остатке товара на складе
Вывод уведомления о необходимости закупки в виде всплывающего окна
Выделение товара красным цветом в таблице при выполнении функции "Отображение информации о каждой позиции товаров"
itemize

%%%%%%%%%%%%%%%%%%%%%%%%%%%%%%%%%%%%%%%

% Отображение всех клиентов компании
% Отображение и изменение подробной информации о клиенте
% Отображение и изменение информации о договорах и других документах клиента
% Внесение и отображение информации о сервисном обслуживании оборудования клиента

\subsubsection{Описание требований к функции ''Отображение всех клиентов компании''}
Функция "Отображение всех клиентов компании" должна отображать полный список клиентов компании в графическом интерфейсе. Пользователь должен иметь возможность перейти к конкретному клиенту, нажав на строку с его ФИО в списке. Пользователь должен иметь возможность осуществить поиск по ФИО, отсортировать таблицу по алфавиту. Пользователь должен иметь возможность добавить нового клиента.

Для этого необходимо выполнить следующие подзадачи:
itemize
Получение строки поиска или параметра сортировки
Отображение списка всех клиентов компании в интерфейсе
Переход к функции "Отображение и изменение подробной информации о клиенте, о всех выполненных заказов клиента", если пользователь нажимает на строку конкретного клиента
Переход к функции "Заполнение информации о новом клиенте или контактном лице", если пользователь добавляет нового клиента
itemize

\subsubsection{Описание требований к функции ''Отображение и изменение подробной информации о клиенте''}
Функция "Отображение и изменение подробной информации о клиенте" должна предоставлять пользователю наиболее полную информацию о клиенте в удобном графическом интерфейсе. Пользователь должен иметь возможность изменить информацию о клиенте. Пользователь должен иметь возможность добавить новый заказ или покупку клиента.

Для этого необходимо выполнить следующие подзадачи:
itemize
Отображение и изменение ФИО, адреса, списка контактных лиц и номеров телефонов клиента
Отображение всех заказов клиента
Отображение покупок клиента
Отображение и изменение заметок и дополнительной информации о клиенте
Переход к функции "Регистрация нового заказа клиента", если пользователь добавляет новый заказ
Переход к функции "Внесение информации о поступлении и отгрузке товаров со склада", если пользователь добавляет новую покупку клиента
Сохранение информации в базе данных
itemize

\subsubsection{Описание требований к функции ''Отображение и изменение информации о договорах и других документах клиента''}
Функция "Отображение и изменение информации о договорах и других документах клиента" должна отображать информацию подписанных с клиентом документах, об их статусе. Пользователь должен иметь возможность изменять информацию о документах и создавать новые документы, а также автоматически создавать документы из шаблонов.

Для этого необходимо выполнить следующие подзадачи:
itemize
Отображение информации о подписанных с клиентом документах
Изменение информации о каждом отдельном документе
Добавление записи о новом документе
Создание документа из шаблона
Сохранение информации в базе данных
itemize

\subsubsection{Описание требований к функции ''Внесение и отображение информации о сервисном обслуживании оборудования клиента''}
Функция "Внесение и отображение информации о сервисном обслуживании оборудования клиента" должна отображать информацию о дате последнего сервисного обслуживания оборудования и уведомлять пользователя о приближении следующего планового сервисного обслуживания. Пользователь должен иметь возможность указать, как часто необходимо обслуживать оборудование.

Для этого необходимо выполнить следующие подзадачи:
itemize
Получение информации об установке и последнем обслуживании оборудования
Получение информации о рекомендуемой частоте обслуживания оборудования
Получение информации о следующем напоминании о необходимости обслуживания
Отображение окна с напоминанием о необходимости обслуживания
Ввод результата звонка клиенту по поводу обслуживания оборудования. Перенос даты напоминания или переход к функции "Регистрация нового заказа клиента"
Сохранение информации в базе данных
itemize

%%%%%%%%%%%%%%%%%%%%%%%%%%%%%%%%%%%%%%%


\subsection{Требования к видам обеспечения}

\subsubsection{Требования к математическому обеспечению системы}
Математические методы и алгоритмы, используемые для шифрования/дешифрования данных, а также программное обеспечение, реализующее их, должны быть сертифицированы уполномоченными организациями для использования в государственных органах Российской Федерации.
\subsubsection{Требования к информационному обеспечению системы}
Состав, структура и способы организации данных в системе должны быть определены на этапе технического проектирования. Уровень хранения данных в системе должен быть построен на основе современных СУБД. Для обеспечения целостности данных должны использоваться встроенные механизмы СУБД. Средства СУБД, а также средства используемых операционных систем должны обеспечивать документирование и протоколирование обрабатываемой в системе информации. Структура базы данных должна поддерживать кодирование хранимой и обрабатываемой информации в соответствии с общероссийскими классификаторами (там, где они применимы).\\

Доступ к данным должен быть предоставлен только авторизованным пользователям с учетом их служебных полномочий, а также с учетом категории запрашиваемой информации. Структура базы данных должна быть организована рациональным способом, исключающим единовременную полную выгрузку информации, содержащейся в базе данных системы. Технические средства, обеспечивающие хранение информации, должны использовать современные технологии, позволяющие обеспечить повышенную надежность хранения данных и оперативную замену оборудования (распределенная избыточная запись/считывание данных; зеркалирование; независимые дисковые массивы; кластеризация).\\

В состав системы должна входить специализированная подсистема резервного копирования и восстановления данных. При проектировании и развертывании системы необходимо рассмотреть возможность использования накопленной информации из уже функционирующих информационных систем. Перечень функционирующих информационных систем приведен в разделе 3 настоящего документа.

\subsubsection{Требования к лингвистическому обеспечению системы}
Все прикладное программное обеспечение системы для организации взаимодействия с пользователем должно использовать русский язык.

\subsubsection{Требования к программному обеспечению системы}
При проектировании и разработке системы необходимо максимально эффективным образом использовать ранее закупленное программное обеспечение, как серверное, так и для рабочих станций.

Используемое при разработке программное обеспечение и библиотеки программных кодов должны иметь широкое распространение, быть общедоступными и использоваться в промышленных масштабах. Базовой программной платформой должна являться операционная система MS Windows.

\subsubsection{Требования к техническому обеспечению}
Техническое обеспечение системы должно максимально и наиболее эффективным образом использовать существующие в организации технические средства. В состав комплекса должны входить следующие технические средства:
itemize
Сервер БД;
ПК администратора.
itemize

Сервер и рабочие станции должны быть объединены одной локальной сетью с пропускной способностью не менее 25 Мбит/с.

Требования к техническим характеристикам сервера БД:
itemize
Процессор – Intel Xeon 3 ГГц;
Операционная система – Microsoft Windows Server 2012 или выше;
Объем оперативной памяти – 8 Гб;
Дисковая подсистема – 2 х 150 Гб;
Сетевой адаптер – 25 Мбит/с.
itemize

Требования к техническим характеристикам рабочих станций:
itemize
Процессор – Intel Core i3 1,2 ГГц;
Объем оперативной памяти – 4 Гб;
Объем жесткого диска – 80 Гб;
Операционная система – Windows 7/Windows 10;
Сетевой адаптер – 25 Мбит/с.
itemize


\subsubsection{Требования к метрологическому обеспечению}
Требования к метрологическому обеспечению не предъявляются.

\subsubsection{Требования к организационному обеспечению}
Организационное обеспечение системы должно быть достаточным для эффективного выполнения персоналом возложенных на него обязанностей при осуществлении автоматизированных и связанных с ними неавтоматизированных функций системы.

Заказчиком должны быть определены должностные лица, ответственные за: обработку информации АИС, администрирование АИС, обеспечение безопасности информации АИС, управление работой персонала по обслуживанию АИС.

К работе с системой должны допускаться сотрудники, имеющие навыки работы на персональном компьютере, ознакомленные с правилами эксплуатации и прошедшие обучение работе с системой.

\subsubsection{Требования к методическому обеспечению}
Состав нормативно-правового и методического обеспечения системы должны входить следующие законодательные акты, стандарты и нормативы:
\begin{enumerate}
Федеральный закон "Об информации, информационных технологиях и о защите информации" от 27.07.2006 \No 149-ФЗ;
Устав компании.
\end{enumerate}

\section{Состав и содержание работ по созданию (развитию) системы}
\begin{tabular}{|c|p{8cm}|p{6cm}|}
\hline
Этап & Содержание работ & Результат работ \\
\hline
1 & Разработка рабочей документации АИС "Мастер CRM". & Рабочая документация АИС.\\
\hline
2 & Создание подсистем формирования оценочного листа, формирования сопроводительных актов, формирования отчетности. & Программное обеспечение указанных подсистем.\\
\hline
3 & Тестирование и отладка программного обеспечения АИС. & Отчет о тестировании АИС, программное обеспечение АИС.\\
\hline
4 & Разработка руководства пользователя АИС. & Руководство пользователя АИС.\\
\hline
5 & Приемочные испытания АИС. & Акт приемочной комиссии АИС "Мастер CRM".\\
\hline

\end{tabular}


\section{Порядок контроля и приемки системы}
\subsection{Виды, состав, объем и методы испытаний системы}
Виды, состав, объем, и методы испытаний подсистемы должны быть изложены в программе и методике испытаний АИС "Мастер CRM", разрабатываемой в составе рабочей документации.
\subsection{Общие требования к приемке работ по стадиям}
Сдача-приемка осуществляется комиссией, в состав которой входят представители Заказчика и исполнителя. По результатам приемки подписывается акт приемочной комиссии. Все создаваемые в рамках настоящей работы программные изделия (за исключением покупных) передаются Заказчику как в виде готовых модулей, так и в виде исходных кодов, предоставляемых в электронной форме на стандартном машинном носителе (например, на USB-флеш-накопителе).
\subsection{Статус приемочной комиссии}
Статус приемочной комиссии определяется Заказчиком до проведения испытаний.

\section{Требования к составу и содержанию работ по подготовке объекта автоматизации к вводу системы в действие}
В ходе выполнения проекта на объекте автоматизации требуется выполнить работы по подготовке к вводу системы в действие.

При подготовке к вводу в эксплуатацию АИС "Мастер CRM" Заказчик должен обеспечить выполнение следующих работ:
\begin{enumerate}
Определить подразделение и ответственных должностных лиц, ответственных за внедрение и проведение опытной эксплуатации АИС "Мастер CRM";
Обеспечить присутствие пользователей на обучении работе с системой, проводимом исполнителем;
Обеспечить соответствие помещений и рабочих мест пользователей системы с требованиями, изложенными в настоящем документе;
Обеспечить выполнение требований, предъявляемых к программно-техническим средствам, на которых должно быть развернуто программное обеспечение АИС "Мастер CRM";
Совместно с исполнителем подготовить план развертывания системы на технических средствах Заказчика;
Провести опытную эксплуатацию АИС "Мастер CRM".
\end{enumerate} 

Требования к составу и содержанию работ по подготовке объекта автоматизации к вводу системы в действие, включая перечень основных мероприятий и их исполнителей должны быть уточнены на стадии подготовки рабочей документации и по результатам опытной эксплуатации.

\section{Требования к документированию}

Документы должны быть представлены в бумажном виде (оригинал) и на носителе (копия). Исходные тексты программ --- только на носителе (оригинал).
Все документы должны быть оформлены на русском языке. Состав документов на общее программное обеспечение, поставляемое в составе АИС, должен соответствовать комплекту поставки компании --- изготовителя.

Подлежащие разработке документы:
\begin{enumerate}
Документ «Описание предметной области» согласно со стандартом оформления документации ГОСТ 2.105-95
Документ «Техническое задание» согласно ГОСТ 34.602.2020
Документ «Руководство пользователя» согласно РД 50-34.698-90 п.3.4.
Документ «Программа и методики испытаний» согласно ГОСТ 19.301-7
Документа «Паспорт» согласно ГОСТ 50-34.698-90 п.2.8.
\end{enumerate}

К видам программной документации относят документы, содержащие сведения, необходимые для разработки, изготовления, сопровождения и эксплуатации программ:
\begin{enumerate}
Спецификация (состав программы и документации на нее)
Ведомость держателей подлинников (перечень предприятий, на которых хранят подлинники программных документов)
Текст программы (запись программы с необходимыми комментариями)
Описание программы (сведения о логической структуре и функционировании программы)
Программа и методика испытаний (требования, подлежащие проверке при испытании программы, а также порядок и методы их контроля)
Техническое задание
Пояснительная записка (схема алгоритма функционирования программы, а также обоснование принятых технических решений)
Эксплуатационные документы (сведения для обеспечения функционирования и эксплуатации программы)
\end{enumerate}

\newpage
\section{Источники разработки}
Техническое задание разработано на основе следующих документов:\\

Учебники, учебные пособия и другие материалы:
\begin{enumerate}
Автоматизация управления предприятием. Модели и методы исследования предприятия: учебное пособие для студентов вузов / Д. П. Косицын, И. М. Шабалина; М-во образования и науки Рос. Федерации, Федер. гос. бюджет. образоват. учреждение высш. образования Петрозавод. гос. ун-т. – Петрозаводск: Издательство ПетрГУ, 2016 – 56 с. (дата обращения: 03.11.2024)
ГОСТ 34.201-89 Информационная технология (ИТ). Комплекс стандартов на автоматизированные системы. [Электронный ресурс] URL:\\* http://docs.cntd.ru/document/gost-34-201-89 (дата обращения: 27.11.2024)
\end{enumerate}

Нормативные правовые акты:
\begin{enumerate}
Федеральный закон «Об информации, информационных технологиях и о защите информации» от 27.07.2006 \No 149-ФЗ.
Государственные стандарты:
ГОСТ 34.602.2020 «Техническое задание на создание автоматизированной системы»;
ГОСТ РД 50-34.698-90 «Автоматизированные системы. Требования к содержанию документов»;
ГОСТ 19.301-79 «Единая система программной документации (ЕСПД). Программа и методика испытаний. Требования к содержанию и оформлению (с Изменениями \No 1, 2)»
\end{enumerate}

\end{document}
