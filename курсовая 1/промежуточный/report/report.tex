%%% Для сборки выполнить 2 раза команду: pdflatex <имя файла>

\documentclass[a4paper,12pt]{article}

\usepackage{ucs}
\usepackage[utf8x]{inputenc}
\usepackage[russian]{babel}
%\usepackage{cmlgc}
\usepackage{graphicx}
\usepackage{listings}
\usepackage{xcolor}
%\usepackage{courier}

\makeatletter
\renewcommand\@biblabel[1]{#1.}
\makeatother

\newcommand{\myrule}[1]{\rule{#1}{0.4pt}}
\newcommand{\sign}[2][~]{{\small\myrule{#2}\\[-0.7em]\makebox[#2]{\it #1}}}

% Поля
\usepackage[top=20mm, left=30mm, right=10mm, bottom=20mm, nohead]{geometry}
\usepackage{indentfirst}

% Межстрочный интервал
\renewcommand{\baselinestretch}{1.50}


\begin{document}

%%%%%%%%%%%%%%
%%% Титульный лист %%%
%%%                             %%%

\thispagestyle{empty}
\begin{center}


\renewcommand{\baselinestretch}{1}
{\large
{\sc Петрозаводский государственный университет\\
Институт математики и информационных технологий\\
	Кафедра Информатики и Математического Обеспечения
}
}

\end{center}


\vfill

\begin{center}
{\normalsize Промежуточный отчет о научно-исследовательской работе} \\

\medskip

%%% Название работы %%%
	{\Large \sc Создание веб-сервиса для коллективных переводов} \\
\end{center}

\medskip

\begin{flushright}
\parbox{11cm}{%
\renewcommand{\baselinestretch}{1.2}
\normalsize
	Выполнила:\\

студентка 3 курса группы 22305 С. Э. Зименкова
\begin{flushright}
	\sign[подпись]{4cm}
\end{flushright}




Научный руководитель:\\
%%% степень, звание ФИО научного руководителя %%%
% Первый руководитель 
Д. Б. Чистяков \\
Оценка руководителя:
\begin{flushright}
\sign{4cm}\\
\sign[подпись]{4cm}
\end{flushright}


% Второй руководитель 
% \textcolor{red}{И. О. Фамилия, ученая степень, ученое звание} \\
% \begin{flushright}
% \sign[подпись]{4cm}
% \end{flushright}
}
\end{flushright}

\vfill

\begin{center}
\large
    Петрозаводск --- 2023
\end{center}

%%% Титульный лист %%%
%%%                             %%%
%%%%%%%%%%%%%%


%%%%%%%%%%%%%
%%% Содержание    %%%
%%%                          %%%

\newpage

\tableofcontents

%%% Содержание    %%%
%%%                          %%%
%%%%%%%%%%%%%


%%%%%%%%%%%
%%% Введение %%%
%%%                  %%%

\newpage
\section*{Введение}
\addcontentsline{toc}{section}{Введение}

Перевод больших массивов текста зачастую является трудоемкой задачей, требующей большого количества времени. Задача усложняется, если необходимо перевести художественное произведение, научную работу, текст выступления на конференции или справочные материалы, содержащие большое количество специальной лексики. Процесс перевода становится более эффективным, если он выполняется коллективом профессионалов, выполняющих разные роли: помимо самих переводчиков в работе принимают участие редакторы и руководители проекта, которым нужно разделять обязанности между работниками, согласовывать процесс и утверждать прошедшие редакцию главы. \\

Профессиональный перевод не является единственной возможной формой работы над текстом на иностранном языке. Не все заказчики могут позволить себе оплатить профессиональный перевод значительного по объему текста. Примером подобных текстов могут выступать интерфейс ПО и компьютерных игр, узконаправленная литература, конспекты лекций и конференций, субтитры к видеоматериалам, находящимся в открытом доступе. Такую работу часто выполняют волонтеры, не все из которых обладают достаточными навыками для выполнения профессионального перевода. В проектах, связанных с любительским переводом, помимо синхронизации и распределения частей текста между участниками, необходимо проверять качество работы и выбирать наилучшие варианты перевода, так как волонтеры и любители не несут достаточной ответственности за свою работу и не могут гарантировать его верность. \\

Целью данной работы является разработка front-end составляющей предложенного инструмента. Инструмент будет реализован в форме веб-сайта, позволяющего пользователям:
\begin{itemize}
	\item Создавать новые проекты и принимать участие в существующих проектах:
	\begin{description}
		\item[-] Пользователь может создать проект, присвоить ему свойства, добавить описание, сделать проект открытым или закрытым для других пользователей, приглашать других пользователей стать участниками проекта, разрешать другим пользователям становиться участниками проекта.
		\item[-] Пользователь может искать открытые проекты, оставлять заявку на участие в них и принимать участие после одобрения заявки руководителем проекта; получать доступ по ссылке к закрытым проектам и принимать в них участие после одобрения заявки руководителем проекта; принимать участие в проектах, в которые был приглашен.
	\end{description}
	\item Разбивать исходный текст на отдельные главы, а главы на отдельные фрагменты:
	\begin{description}
		\item[-] Пользователь может использовать автоматическое разбиение на фрагменты по переносам строки или вручную задавать разбиение текста.
	\end{description}
	\item Распределять роли между участниками проекта:
	\begin{description}
		\item[-] В проекте существуют роли, которые присваиваются участникам. В зависимости от роли участник может выполнять различные функции и действия. Например, роль редактора позволяет участнику утверждать верный вариант перевода и вносить правки в итоговый результат, роль руководителя проекта позволяет участнику назначать другие роли и выполнять все доступные действия.
	\end{description}
	\item Распределять работу над фрагментами между участниками проекта:
	\begin{description}
		\item[-] В проекте есть возможность создавать связь между фрагментом текста и участником, выполняющим перевод этого фрагмента. Эта функция позволяет по необходимости исключить выполнение перевода двумя разными участниками, что повышает эффективность за счет сокращения потраченного на выполнение лишней работы времени.
	\end{description}
	\item Предлагать варианты перевода каждого фрагмента:
	\begin{description}
		\item[-] Участник проекта может добавить вариант перевода этого фрагмента. Все предложенные варианты перевода доступны для прочтения другим участникам проекта.
	\end{description}
	\item Рецензировать и редактировать перевод:
	\begin{description}
		\item[-] Путем голосования наиболее верным вариантом считается тот, который получил наибольшее количество голосов от участников. Между вариантами с одинаковым количеством голосов выбирается тот, который был составлен позже.
		\item[-] Участники с более высокими ролями (редактор, руководитель проекта) могут выбрать лучший на свой взгляд вариант перевода вне зависимости от итогов голосования. Решение участника с более высокой ролью превалирует над решением участника с менее высокой ролью.
	\end{description}
	\item Сохранять итоговый результат:
	\begin{description}
		\item[-] Итоговый результат может быть получен в различном виде. Кроме формата полученного файла можно также выбрать, какой перевод будет вставлен в итоговый файл. Эти параметры будут настраевыми при экспорте перевода.
	\end{description}
\end{itemize}

На данном этапе сформулированы следующие задачи:
\begin{itemize}
	\item изучить JavaScript-библиотеку React и работу с API с помощью этой библиотеки;
	\item на основе прототипов веб-страниц разработать динамические страницы веб-сайта;
	\item реализовать заполнение контента веб-страниц с помощью запросов к API;
	\item реализовать отправку данных в базу данных с помощью запросов к API.
\end{itemize}

%%% Введение %%%
%%%                  %%%
%%%%%%%%%%%


%%%%%%%%%%%
%%% Раздел 1  %%%
%%%                  %%%

\newpage
\section{Описание инструментов и технологий}

Для front-end разработки будут использованы следующие инструменты:
\begin{itemize}
	\item JavaScript-библиотека -- \texttt{React};
	\item вспомогательная библиотека -- \texttt{React Bootstrap};
\end{itemize}

Прежде для создания front-end составляющей были рализованы прототипы страниц веб-сайта на языке разметки HTML5 с помощью библиотеки Bootstrap. В этой работе рассматривается разработка динамических страниц с помощью библиотек React и React Bootstrap. Прототипы, созданные с помощью языка языка разметки, более удобны, чем прототипы, созданные с помощью графических редакторов, т. к. разработка динамических страниц даже с помощью другого языка разметки (React использует разметку-расширение JSX, а не HTML) таким образом происходит намного быстрее. Библиотека React Bootstrap также легко портируется с HTML-разметки в код React-приложения. React Bootstrap значительно облегчает работу с визуальными компонентами, т. к. предоставляет большое количество настраиваемых компонент.\\
 
%%% Раздел 1  %%%
%%%                  %%%
%%%%%%%%%%%

\end{document}