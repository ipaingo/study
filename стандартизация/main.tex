\newcommand*{\No}{\textnumero}
\documentclass[russian, utf8, 12pt,pointsubsection,floatsubsection]{eskdtext}
\usepackage[russian]{babel} 

	\ESKDclassCode{ТП}
	\ESKDtitle{Автоматизированная система подачи заявления о выдаче водительских прав}
	\ESKDdocName{Программа и методика испытаний}
	\ESKDsignature{65698922.425120.022.И3.01.01}
	\ESKDcolumnII{65698922.425120.022.И3.01.01}
	\ESKDcolumnI{Программная документация. ООО «Название». РП}

	\ESKDgroup{ООО <<Название>>}
	%\ESKDauthor{Фамилия И.О.}
	\ESKDtitleAgreedBy{Директор ООО <<Название>>}{Фамилия И.О.}
	\ESKDtitleDesignedBy{Зам. директора <<Название>>}{Фамилия И.О.}

	\ESKDtitleApprovedBy{Директор ООО «Название»}{Фамилия И.О.} 
	\ESKDtitleAgreedBy{Главный технолог ООО «Название»}{Фамилия И.О.} 


 	\ESKDdate{2022/11/30} 

% \title{Text}
% \author{ }
% \date{September 2021}

\begin{document}

	% титульный лист 
	\maketitle 

	% оглавление 
	\scriptsize
	\setcounter{tocdepth}{4}
	\tableofcontents
	\normalsize
	\newpage
	
% "Программа и методика испытаний" комплекса средств автоматизации проектирования на этапе опытного функционирования предназначена для установления технических данных, подлежащих проверке при испытании компонентов АС и комплекса средств автоматизации проектирования, а также порядок испытаний и методы их контроля. 
% "Программа и методика испытаний" системы (подсистемы) на этапе опытного функционирования предназначена для установления данных, обеспечивающих получение и проверку проектных решений, выявление причин сбоев, определение качества работ, показателей качества функционирования системы (подсистемы), проверку соответствия системы требованиям техники безопасности, продолжительность и режим испытаний.
% Программы испытаний должны содержать перечни конкретных проверок (решаемых задач), которые следует осуществлять при испытаниях для подтверждения выполнения требований ТЗ, со ссылками на соответствующие методики (разделы методик) испытаний.
% Перечень проверок, подлежащих включению в программу испытаний, включает: 
% 1) соответствие системы ТЗ; 
% 2) комплектность системы; 
% 3) комплектность и качество документации; 
% 4) комплектность, достаточность состава к качество программных средств и программной документации; 
% 5) количество и квалификация обслуживающего персонала; 
% 6) степень выполнения требований функционального назначения системы; 
% 7) контролепригодность системы; 
% 8) выполнение требований техники безопасности, противопожарной безопасности, промышленной санитарии, эргономики; 
% 9) функционирование системы с применением программных средств.
%  Описание методов испытаний системы по отдельным показателям рекомендуется располагать в той же последовательности, в которой эти показатели расположены в технических требованиях.
% Программа испытаний содержит разделы: 
% 1) объект испытаний; 
% 2) цель испытаний; 
% 3) общие положения; 
% 4) объем испытаний; 
% 5) условия и порядок проведения испытаний; 
% 6) материально-техническое обеспечение испытаний; 
% 7) метрологическое обеспечение испытаний; 
% 8) отчетность.
% В документ включают приложения. 
% В зависимости от особенностей систем допускается объединять или исключать отдельные разделы при условии изложения их содержания в других разделах программы испытаний, а также включать в нее дополнительные разделы (при необходимости).
	
	
	
	
\section{ОБЪЕКТ ИСПЫТАНИЙ}
% В разделе "Объект испытаний" указывают: 
% 1) полное наименование системы, обозначение; 
% 2) комплектность испытательной системы.

\subsection{Полное наименование системы и ее условное обозначение}
Полное наименование системы: Автоматизированная система подачи заявления о выдаче водительских прав "ГлавПрав".\\
Краткое наименование системы: АС "ГлавПрав".\\ 

\subsection{Комплектность испытательной системы}
Систему регистрации заявок можно разделить на 2 подсистемы:
\begin{itemize}
    \item подсистема пользователя клиента;
    \item подсистема пользователя, который обрабатывает услуги (оператор)
\end{itemize}

\section{ ЦЕЛЬ ИСПЫТАНИЙ}
% В разделе "Цель испытаний" указывают конкретные цели и задачи, которые должны быть достигнуты и решены в процессе испытаний. 
Целью проведения испытаний является последовательная проверка соответствия АС "ГлавПрав" требованиям Технического задания.\\
В частности работоспособность функций:
\begin{itemize}
    \item ''Получение данных клиента для заполнения заявки'';
    \item ''Отправка заявления в подразделение'';
\end{itemize}

\section{ОБЩИЕ ПОЛОЖЕНИЯ}

% В разделе "Общие положения" указывают: 
% 1) перечень руководящих документов, на основании которых проводят испытания; 
% 2) место и продолжительность испытаний; 
% 3) организации, участвующие в испытаниях; 
% 4) перечень ранее проведенных испытаний; 
% 5) перечень предъявляемых на испытания документов, откорректированных по результатам ранее проведенных испытаний. 
% 4.1 Документы, на основания которых проводят испытания
% 4.2 Место и продолжительность испытаний
% 4.3 Организации, участвующие в испытаниях
% 4.4 Испытания, проведенные ранее
% 4.5 Предъявляемые на испытания документы

\subsection{Документы, на основания которых проводят испытания}

Настоящая программа и методика испытаний разработана в соответствии со следующими документами: 
\begin{itemize}
    \item ГОСТ 19.201-78 Техническое задание на разработку; 
    \item Техническое задание АС "ГлавПрав";
    \item ГОСТ 19.301-79 Программа и методика испытаний.
\end{itemize}

\subsection{Место и продолжительность испытаний}
Испытания проводятся на территории Исполнителя.\\

Плановый срок начала работ по созданию Аввтоматизированной система регистрации заявлений на получение прав "ГлавПрав" для подразделения ГИБДД города Петрозаводск – 01 декабря 2022 года.\\

Плановый срок окончания работ по созданию Автоматизированной система регистрации заявлений на получение прав "ГлавПрав" для подразделения ГИБДД города Петрозаводск – 30 июля 2023 года.\\

\subsection{Организации, участвующие в испытаниях}
Разработчиком системы является ООО "Kitty-cat".\\

Контактные данные директора Першина О. В.: 
\begin{itemize}
    \item телефон: +7(953)530-46-20 
    \item электронная почта: catola2001@gmail.com
\end{itemize}

Адрес разработчика: 185030 Россия, Петрозаводск, проспект Ленина, 33\\

\subsection{Испытания, проведенные ранее}
Ранее испытания системы не проводились.\\

\subsection{Предъявляемые на испытания документы}
На испытания предъявляются следующие документы:
\begin{itemize}
    \item Государственный контракт №1/11-11-11-001 от 11.11.2022 года на выполнение работ по выполнению первого этапа работ по созданию Автоматизированной система регистрации заявлений о выдаче водительских прав "ГлавПрав" для подразделения ГИБДД города Петрозаводск;
    
    \item Постановление Правительства РФ от 01 января 2000 г. N 11.11 «О федеральной целевой программе "Электронные документы (2020 - 2024 годы)»;
    \item Концепция информатизации федерального агентства "Государственные стандарты" на 2020-2030 годы.
    \item Техническое задание АС "ГлавПрав";
    \item Общее описание системы АС "ГлавПрав";
    \item Руководство пользователя АС "ГлавПрав";
    \item Настоящая Программа и методика предварительных испытаний.
\end{itemize}

\section{ОБЪЕМ ИСПЫТАНИЙ}

% В разделе "Объем испытаний" указывают: 
% 1) перечень этапов испытаний и проверок, а также количественные и качественные характеристики, подлежащие оценке; 
% 2) последовательность проведения и режима испытаний; 
% 3) требования по испытаниям программных средств; 
% 4) перечень работ, проводимых после завершения испытаний, требования к ним, объем и порядок проведения. 

\subsection{Этапы испытаний}
Испытания проводятся в один этап.\\
В ходе испытаний выполняются проверки: 
\begin{enumerate}
%    \item проверка комплектности программного обеспечения;	
    \item проверка комплектности документации;
    \item проверка качества документации;
    \item проверка соответствия функциональности программного обеспечения АС "ГлавПрав"\\
    требованиям Технического задания;
\end{enumerate}


    
\subsubsection{Проверка комплектности документации}
В соответствии с требованиями Заказчика поставляемая документация должна включать следующие документы:

\begin{tabular}{|r|l|}
\hline \hline
Наименование экземпляра & Поставляемое количество (шт.) \\
\hline \hline
Техническое задание & 1 \\
\hline
Руководство пользователя & 1 \\
\hline
Общее описание системы & 1 \\
\hline 
Программа и методика испытаний & 1 \\
\hline
\end{tabular}

\subsubsection{Проверка качества документации}
Каждый документ оценивается набором оценок по пятибалльной шкале в соответствии с требованиями:
\begin{itemize}
    \item Уровень технического исполнения документа, единство форматирования документа 
    \item Понятность изложения материала
    \item Полнота изложения, необходимая для эксплуатации Системы (для эксплуатационных документов)
    \item Наличие всех рисунков, таблиц и формул
    \item Правильность использования терминов, четкость формулировок и описаний
    \item Отсутствие неправильных ссылок, в т.ч. и на внешние документы
    \item Отсутствие незаконченных разделов, абзацев, предложений
    \item Отсутствие противоречий
\end{itemize}
 
 Итоговая оценка документа рассчитывается как среднее арифметическое оценок за каждый пункт.
 
\subsubsection{Проверка соответствия функциональности программного обеспечения АС "ГлавПрав" требованиям Технического задания}
Проверка соответствия функциональности предъявленных на испытания подсистем требованиям технического задания осуществляется согласно Приложению А «Методика проведения испытаний.Контрольные примеры».


\subsection{Последовательность проведения испытаний}
Испытания проводятся в порядке, указанном в п. 4.1 данного документа.

\subsection{Требования по испытаниям программных средств}
Объем и методика выполнения проверок приведены в Приложении А «Методика проведения испытаний.Контрольные примеры». 

\subsection{Работы по завершении испытаний}
Результаты испытаний, включая все сведения об обнаруженных сбоях, ошибках в настройках подсистем в ходе проведения испытаний, оформляются Протоколом испытаний. По итогам испытаний принимается решение о готовности ввода подсистем в опытную эксплуатацию. \\

Необходимый объем и порядок проведения работ после завершения испытаний должен быть разработан организациями, участвующими в испытаниях, и занесен в Протокол испытаний. 


\section{УСЛОВИЯ И ПОРЯДОК ПРОВЕДЕНИЯ ИСПЫТАНИЙ}

% В разделе "Условия и порядок проведения испытаний" указывают: 
% 1) условия проведения испытаний; 
% 2) условия начала и завершения отдельных этапов испытаний; 
% 3) имеющиеся ограничения в условиях проведения испытаний; 
% 4) требования к техническому обслуживанию системы; 
% 5) меры, обеспечивающие безопасность и безаварийность проведения испытаний; 
% 6) порядок взаимодействия организаций, участвующих в испытаниях; 
% 7) порядок привлечения экспертов для исследования возможных повреждений в процессе проведения испытаний; 
% 8) требования к персоналу, проводящему испытания, и порядок его допуска к испытаниям.
\subsection{Условия проведения испытаний}
Испытания должны проводиться в нормальных климатических условиях (ГОСТ 22261-94), характеризующихся следующими параметрами:
\begin{enumerate}
    \item температура окружающего воздуха, °С — 20 ± 5;
    \item относительная влажность, \% — 30...80;
    \item атмосферное давление, кПа — 84...106;
    \item частота питающей электросети, Гц — 50 ± 0,5;
    \item напряжение питающей сети переменного тока, В — 220 ± 4.
\end{enumerate}

При проведении испытаний результаты каждой проверки записываются в Протокол испытаний. Испытания проводятся в соответствии с данной Программой и методикой испытаний.

\subsection{Ограничения в условиях проведения испытаний}
Климатические условия эксплуатации, при которых должны обеспечиваться заданные характеристики, должны удовлетворять требованиям, предъявляемым к техническим средствам в части условий их эксплуатации.

\subsection{Требования к техническому обслуживанию системы}
Техническое обслуживание АС "ГлавПрав" включает в себя комплекс операций, направленных на обеспечение штатного функционирования АС. В процессе проведения предварительных испытаний техническое обслуживание АС "ГлавПрав" обеспечивают специалисты Исполнителя.

\subsection{Меры по обеспечению безопасности испытаний}
При проведении испытаний Заказчик должен обеспечить соблюдение требований безопасности, установленных ГОСТ 12.2.007.0–75, ГОСТ 12.2.007.3–75, «Правилами техники безопасности при эксплуатации электроустановок потребителей», и «Правилами технической эксплуатации электроустановок потребителей».

\subsection{Порядок взаимодействия организаций}
Сдача-приемка осуществляется комиссией, в состав которой входят представители Заказчика и Исполнителя. По результатам приемки подписывается акт приемочной комиссии.\\

Все создаваемые в рамках настоящей работы программные изделия (за исключением
покупных) передаются Заказчику, как в виде готовых модулей, так и в виде исходных кодов,
представляемых в электронной форме на стандартном машинном носителе.\\

Заказчик совместно с Исполнителем проводят все подготовительные мероприятия для проведения испытаний на объекте Заказчика, а также проводят испытания в соответствии с настоящей Программой и методикой испытаний.\\

Заказчик осуществляет контроль проведения испытаний, а также документирует ход проведения проверок в Протоколе испытаний.\\

Статус приемочной комиссии определяется Заказчиком до проведения испытаний.\\
\subsection{Порядок привлечения экспертов}
Заказчик может привлекать экспертов на всех этапах испытаний.

\subsection{Требования к персоналу}
Персонал, проводящий испытания, должен обладать навыками работы с персональными компьютерами, операционными системами Microsoft Windows и быть ознакомлен с эксплуатационной документацией на представленные подсистемы.\\


\section{МАТЕРИАЛЬНО-ТЕХНИЧЕСКОЕ ОБЕСПЕЧЕНИЕ ИСПЫТАНИЙ}

% В разделе "Материально-техническое обеспечение испытаний" указывают конкретные виды материально-технического обеспечения с распределением задач и обязанностей организации, участвующих в испытаниях.
% 8 МЕТРОЛОГИЧЕСКОЕ ОБЕСПЕЧЕНИЕ ИСПЫТАНИЙ
% В разделе "Метрологическое обеспечение испытаний" приводят перечень мероприятий по метрологическому обеспечению испытаний с распределением задач и ответственности организаций, участвующих в испытаниях, за выполнение соответствующих мероприятий.
Состав и структура привлекаемых для испытаний технических средств\\
Техническая характеристика серверов БД:
\begin{itemize}
\item Процессор – 2 х Intel Xeon 3 ГГц;
\item Объем оперативной памяти – 16 Гб;
\item Дисковая подсистема – 4 х 146 Гб;
\item Устройство чтения компакт-дисков (DVD-ROM);
\item Сетевой адаптер – 100 Мбит.
\end{itemize}
Техническая характеристика серверов приложений:
\begin{itemize}
\item Процессор – 2 х Intel Xeon 3 ГГц;
\item Объем оперативной памяти – 8 Гб;
\item Дисковая подсистема – 4 х 146 Гб;
\item Устройство чтения компакт-дисков (DVD-ROM);
\item Сетевой адаптер – 100 Мбит.
\end{itemize}
Техническая характеристика  веб сервера:
\begin{itemize}
\item Процессор – 2 х Intel Xeon 3 ГГц;
\item Объем оперативной памяти – 16 Гб;
\item Дисковая подсистема – 4 х 146 Гб;
\item Устройство чтения компакт-дисков (DVD-ROM);
\item Сетевой адаптер – 100 Мбит.
\end{itemize}
Техническая характеристика ПК пользователя и ПК администратора:
\begin{itemize}
\item Процессор – Intel Pentium 1.5 ГГц;
\item Объем оперативной памяти – 256 Мб;
\item Дисковая подсистема – 40 Гб;
\item Устройство чтения компакт-дисков (DVD-ROM);
\item Сетевой адаптер – 100 Мбит.
\end{itemize}

\section{ОТЧЕТНОСТЬ}
% В разделе "Отчетность" указывают перечень отчетных документов, которые должны оформляться в процессе испытаний и по их завершению, с указанием организаций и предприятий, разрабатывающих, согласующих и утверждающих их, и сроки оформления этих документов.
% К отчетным документам относят акт и отчет о результатах испытаний, акт технического состояния системы после испытаний.
В процессе проведения испытаний и по их завершению комиссией оформляются:

\begin{itemize}
    \item Протокол предварительных испытаний АС "ГлавПрав", включающий в себя отчеты о результатах проверок по пунктам Программы и методики испытаний, а также все отметки о недоработках и замечания;
    \item Акт технического состояния системы после испытаний;
    \item Акт ввода Системы в опытную эксплуатацию.
\end{itemize}
Вышеперечисленные документы должны быть подготовлены и представлены на согласование не позднее 7 рабочих дней после завершения испытаний.\\
Протокол испытаний составляется в двух экземплярах.\\
Первый экземпляр подписанного Протокола испытаний остается у Заказчика. Второй экземпляр документов передается Исполнителю.\\

\section{ПРИЛОЖЕНИЕ А}
% В приложения включают перечень методик испытаний, математических и комплексных моделей, применяемых для оценки характеристик системы.
\subsection{Начальные условия}
Строка подключения: http://glavprav.ru \\
В качестве рабочей среды для выполнения сценариев работы АС "ГлавПрав" приемной комиссией может быть выбран любой из нижеперечисленных браузеров: 
\begin{enumerate}
    \item Google Chrome;
    \item Firefox Mozilla;
    \item Opera.
\end{enumerate}

\subsection{Программа испытаний}
Общий перечень проверок АС "ГлавПрав" приведен в таблице А.1.\\

Таблица А.1  Перечень проверок АС "ГлавПрав"\\
\begin{tabular}{|r|l|}
\hline 
Номер и наименование проверки функции & Раздел ТЗ\\
\hline \hline
Выбрать позицию меню из каталога & 4.2.1 \\
\hline
Отправка заявления в
подразделение & 4.2.2 \\
\hline
\end{tabular}

\subsection{Контрольные примеры}
\point{Получение данных клиента для заполнения заявки}\\ 

Предусловие: пользователь открыл главную(начальную) страницу сайта АС "ГлавПрав". Последовательность действий для выполнения контрольного примера приведена в Таблице А.2\\

Таблица А.2 Внести данные для формирования заявления\\
\begin{tabular}{| l |  p{8cm} | p{8cm} |}
\hline 
№  &  Выполняемые действия & Ожидаемый результат\\
\hline \hline
1  & Нажимает на кнопку "Создать заявление на выдачу прав" & Открылась форма заполнения заявления \\
\hline
2  & Нажимает на каждую строку информации и заполняет свои данные & Данные заполняются и сохраняются при переходе к следующей строке\\
\hline
3  & По окончанию заполнения данных нажимает на кнопку "Отправить заявление в подразделение" & Открывается уведомление об успешном формировании заявления и список возможных подразделений для отправки\\
\hline
\end{tabular}
\\


\point{Отправка заявления в подразделение}
\\
Предусловие: пользователь заполнил форму заявления и нажал на кнопку "Отправить заявление в подразделение". Последовательность действий для выполнения контрольного примера приведена в Таблице А.3\\

Таблица А.3 Отправить заявление в подразделение\\
\begin{tabular}{| l |  p{8cm} | p{8cm} |}
\hline 
№  &  Выполняемые действия & Ожидаемый результат\\
\hline \hline
1  & Из представленного списка выбирает нужное подразделение левой кнопкой мыши & Открывается подробная информация об этом подразделении \\
\hline
2  & В подробной информации есть пункт "Имеет возможность принять ваше заявление" & Кнопка "Отправить заявление в это подразделение" активна \\
\hline
3  & Нажимает на кнопку "Отправить заявление в это подразделение" & Появляется уведомлении об успешной отправке заявление в выбранное подразделение \\
\hline
\end{tabular}

\end{document}

